\documentclass{beamer}

\usepackage{beamerthemeCambridgeUS}
\usepackage[frenchb]{babel}
\usepackage[utf8]{inputenc}
\usecolortheme{beaver}

\usepackage{listings}

\setbeamertemplate{navigation symbols}{
\insertframenumber/
\inserttotalframenumber
}

%-------------------------------------------------------------------
\title{Multi-Stencil agnostic descriptive language}
\author{Hélène Coullon}
\institute{INRIA}
%-------------------------------------------------------------------

\begin{document}

% \AtBeginSection[ ]
% {
%  \begin{frame}
%    \tableofcontents[currentsection, hideallsubsections]
%   \end{frame}
% }
\begin{frame}
    \titlepage
\end{frame}

% presentation of the table of contents
\begin{frame}
\frametitle{Table of contents}
    \tableofcontents[hideallsubsections]
\end{frame}

%-------------------------------------------------------------
\section{Introduction}
%-------------------------------------------------------------
\begin{frame}{Introduction}
\begin{block}{What is a stencil ?}
In a mesh-based simulation, it is an explicit numerical schemes involving a neighborhood of an element for its computation
\end{block}

\begin{block}{What are Multi-Stencil Porgrams ?}
A real case numerical simulation is composed of
\begin{itemize}
\item more than one computation involving a neighborhood shape,
\item sometimes more than one stencil shape,
\item boundary condition computations,
\item and additionnal local computations.
\end{itemize}
\end{block}

\end{frame}

\begin{frame}{Introduction}
\begin{block}{What does produce the MS Language ?}
Unlike other solutions MS Language gives
\begin{itemize}
\item an hybrid parallelization structure of the overall numerical simulation
\begin{itemize}
\item data parallelism (domain decomposition),
\item and task parallelism to increase parallelism for large scale parallel machines
\end{itemize}
\item the parallelized structure is made of component which increases code re-use, maintainability and portability
\end{itemize}
\end{block}

\begin{block}{What means agnostic ?}
The language do not need any knowledge on
\begin{itemize}
\item the mesh type
\item the neighborhood shape
\item the underlying programming language used
\end{itemize}
\end{block}
\end{frame}

\begin{frame}{Introduction}
\begin{block}{What means descriptive ?}
The language is based on the textual description of the simulation, using identifiers choosen by the user.
\end{block}

\begin{block}{What are the limitations ?}
\begin{itemize}
\item explicit numerical schemes only
\item for now a single mesh type for a simulation
\end{itemize}
\end{block}
\end{frame}

%-------------------------------------------------------------
\section{Overview}
%-------------------------------------------------------------
\begin{frame}{Overview}
\begin{itemize}
\item Data description
\item Time description
\item Computation description
\end{itemize}
\end{frame}

%-------------------------------------------------------------
\section{Data description}
%-------------------------------------------------------------
\begin{frame}[fragile]{Data description}
\begin{block}{One data per line}
\begin{lstlisting}[basicstyle=\footnotesize]
data:
  name_id,domain_id
\end{lstlisting}
\end{block}
example:\\
\begin{lstlisting}[basicstyle=\footnotesize]
data:
  h,d1
  g,cell
\end{lstlisting}
\end{frame}

%-------------------------------------------------------------
\section{Time description}
%-------------------------------------------------------------
\begin{frame}[fragile]{Time description}
\begin{block}{A single line}
\begin{lstlisting}[basicstyle=\footnotesize]
time:nb_iteration
\end{lstlisting}
\end{block}
example for 500 iterations:\\
\begin{lstlisting}[basicstyle=\footnotesize]
time:500
\end{lstlisting}
\end{frame}

%-------------------------------------------------------------
\section{Computation description}
%-------------------------------------------------------------
\begin{frame}[fragile]{Computation description}
\begin{block}{One computation per line}
\begin{lstlisting}[basicstyle=\footnotesize]
computations:
  type:name_id({read1_id,read2_id,...},
    written_id[,neighborhood_id])
\end{lstlisting}
\end{block}
Three types available:
\begin{itemize}
\item stencil: it is a computation which involves a neighborhood around the element to compute
\item local:  it is a computation which does not involve a neighborhood around the element to compute
\item bound: it is a computation on the physical border of the domain (additional elements)
\end{itemize}
\end{frame}

\begin{frame}[fragile]{Computation description}
\begin{block}{One computation per line}
\begin{lstlisting}[basicstyle=\footnotesize]
computations:
  type:name_id({read1_id,read2_id,...},
    written_id[,neighborhood_id])
\end{lstlisting}
\end{block}
\begin{itemize}
\item read1\_id, read2\_id: data identifiers read by the computation
\item written\_id: data identifier written by the computation (a single one)
\item if type==stencil precise a neighborhood identifier
\end{itemize}
\end{frame}

%-------------------------------------------------------------
\section{Overall example}
%-------------------------------------------------------------
\begin{frame}{Overall example}
\begin{block}{Data}
\begin{itemize}
\item 3 data on a domain called \textit{cell}
\item 2 data on a domain called \textit{edgex}
\item 2 data on a domain called \textit{edgey}
\end{itemize}
\end{block}
\begin{block}{Computations}
\begin{itemize}
\item 4 local computations
\item 2 stencil computations
\end{itemize}
\end{block}
\end{frame}

\begin{frame}[fragile]{Overall example}
\begin{lstlisting}[basicstyle=\footnotesize,frame=single]
data:
  h,cell
  u,cell
  v,cell
  w,edgex
  z,edgey
  a,edgex
  b,edgey
time:500
computations:
  local:c0({h},u)
  stencil:c1({u},v,N1)
  local:c2({v},w)
  local:c3({v},z)
  stencil:c4({w,v},a,N1)
  local:c6({a},b)
\end{lstlisting}
\end{frame}

\end{document}

