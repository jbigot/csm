% %----------------------------------------
% \subsection{Stencil solutions}
% %----------------------------------------

Many specific solutions exist to ease parallel programming of numerical simulations, %~\cite{Trilinos-Overview,petsc-efficient}. %Solutions such as PETSc~\cite{petsc-efficient} and Trilinos~\cite{Trilinos-Overview} are specific frameworks. They are made to solve many kinds of PDEs using well-known methods and solvers which are efficiently parallelized inside the library. Those frameworks also include solvers for implicit numerical schemes, not addressed in this work. However, using those solutions the user has to use and follow existing tools. In other words, it is not possible for a mathematician or a physician to write his own numerical method if it differs from existing ones. Moreover as PETSc and Trilinos are frameworks they do not propose a domain specific language to define computations. The purpose of those solutions are much more specific than our solution.
as for example %DSLs which propose to generate parallel codes from a sequential imperative stencil description. For example, 
PATUS~\cite{citeulike12258902} and Pochoir~\cite{spaaTangCKLL11} which propose a simple description to write stencil codes for general structured meshes, and OP2~\cite{Giles2011} and Liszt~\cite{DeVito2011LDS} for general unstructured meshes. Those solutions can be considered as stencil compilers. Actually they produce optimized (cache tiling, etc.) or parallel (CUDA, OpenCL, etc.) codes for a \emph{single} stencil computation.
Recently, domain specific languages for very specific numerical methods such as ExaSlang~\cite{Schmitt:2014:EDL:2691166.2691171} for multigrid solvers, have also been proposed.%, and OPS~\cite{Reguly:2014:ODS:2691166.2691173} for meshes of meshes (hybrid mesh) have also been proposed. 
All those solutions let the user implement their own stencil codes in a sequential programming style, and produce high performance codes for shared memory architectures and sometimes distributed memory architectures. 
Most stencil compilers, though, handle the parallelization or the optimization of a single stencil kernel, considering that it represents the main computation time of numerical simulations. However, most real case numerical simulation are composed of more than one stencil kernel, involving one or more stencil shapes, and additional local kernels, as explained in Section~\ref{sect:multistencil}.

Other solutions, as SkelGIS library~\cite{CPE:CPE3494} or Global Arrays~\cite{Nieplocha:2006:AAP:1125980.1125985} propose more or less specialized distributed data structures (specific meshes for SkelGIS, general arrays for Global Arrays), which simplifies or totally hide communications and synchronizations between processes.

Halide~\cite{Ragan-Kelley:2013:HLC:2491956.2462176} is a DSL which focusses on the automatic parallelization and optimization of stencil pipelines in image processing. Thus, this work is the closest to MSL. However, at least one important difference can be noticed: the specific domain it is applied on. Actually, image processing is applied onto 2D or 3D grids, while explicit numerical schemes to solve PDEs can be applied on many different meshes (grids, but also unstructured or hybrid meshes). This main difference is the reason why MSL is a descriptive and agnostic language, and why it does not target specific optimizations, as tiling for example.%optimizations capabilities are limited compared to DSLs applied on known data structures (making available tiling optimizations etc.).

The work presented in this paper takes place at a different abstraction level and can be seen as complementary to stencil compilers or distributed data structures. The MSCAC compiler only needs the agnostic description of the overall simulation to generate a parallel pattern of the program. This parallel pattern then needs to be filled with implementation parameters and computation codes.
%. Thus, depending on the underlying distributed data structure used, those computation codes 
Those computation codes could be generated by stencil compilers or could be written using existing distributed data structures (as SkelGIS in this paper). As a result, MSL is not a new contribution to stencil compilers, to tiling optimizations, or to distributed data structures, but proposes a coarse-grain automatic parallelization of an overall multi-stencil program.
Finally, MSL is a case study to show that component-based runtimes could improve maintainability adn flexibility of DSLs with a low impact on performance. 

% %----------------------------------------
% \subsection{Language, control and component models}
% %----------------------------------------

% \textbf{Put it somewhere else !}

% As far as we know, no existing DSL has been transformed to a HPC component assembly until know. This paper is a contribution to study the interest of using component models to increase code re-use, separation of concerns and maintainability in DSLs. Actually, as the code is not generated inside the DSL itself but is built from the assembly of existing components, many advantages can get out of this work. First, the code-reuse from one DSL to another is increased if the components are well studied and defined. For example, it seems that control components introduced in this paper can be used for many different DSLs in many different domains. Second, separation of concerns in the final code is also improved as the generated component assembly split the different functionalities of the application in different components. As a result, some components can be written by computer-scientists, while other are specific to the application, as for example computation components. Finally, the maintainability of the application is also increased, as each functionality is separated. In the final code if a computation component has to be modified by a numerician, intricate part of the code will be unchanged and not seen by the numerician.

% If the study of a DSL dumped to a component assembly has not been studied until now, some component models have already proposed the introduction of time control and data control in the assembly, as for example STCM~\cite{DBLP:confeuropar2008} and XMan~\cite{He12component-baseddesign}. However, both those solutions does not seem to target high performance computing and distributed components.
