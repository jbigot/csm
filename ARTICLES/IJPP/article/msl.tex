From the formalism detailed in the previous section, the Multi-Stencil Language and its grammar can already be given.
This grammar is sufficient to automatically extract parallelism from the program as will be explained in the next section.

\begin{filecontents*}{grammar.txt}
program ::= "mesh:" meshid 
            "mesh entities:" listgroup
            "computation domains:" listcompdom
            "independent:" listinde
            "stencil shapes:" liststencil
            "quantities:" listquantities
            "scalar" listscalar
            listloop

listgroup ::= groupid "," listgroup | groupid
listcompdom ::= compdom listcompdom | compdom
compdom ::= compdomid "in" groupid
listinde ::= inde listinde | inde
inde ::= compdomid "and" compdomid
liststencil ::= stencil liststencil | stencil
stencil ::= stencilid "from" groupid "to" groupid
listquantities ::= quantity listquantities |  quantity
quantity ::= groupid listquantityid
listquantityid ::= quantityid "," listquantityid |  quantityid
listscalar ::= scalarid "," listscalar | scalarid
listloop ::= loop listloop | loop
loop ::=  "time:" iteration
          "computations:" listcomp
iteration ::= num | scalar
listcomp ::= comp listcomp |  comp
comp ::= written "=" compid "(" listread ")"
written ::= quantityid "[" compdomid "]" | scalar
listread ::= dataread listread | dataread
dataread ::= quantityid "[" stencilid "]" |  quantityid | scalar
\end{filecontents*}

\begin{figure}[hbt]
  \hspace{5mm}
  \begin{minipage}[!h]{0.98\textwidth}
    {\lstinputlisting[basicstyle=\small,mathescape,frame=single,language=Python,numbers=left]{grammar.txt}}   
    \caption{Grammar of the Multi-Stencil Language. \label{fig:grammar}}
  \end{minipage}
\end{figure}

The grammar of the Multi-Stencil Language is given in Figure~\ref{fig:grammar} and an example is provided in Figure~\ref{fig:exmsl}.
A Multi-Stencil program is composed of eight parts.

\begin{filecontents*}{exmsl.txt}
mesh : cart
mesh entities : cell, edgex
computation domains :
  d1 in cell
  d2 in edgex
independent :
  d1 and d2
stencil shapes : 
  ncc from cell to cell
  nce from cell to edgex
  nec from edgex to cell
quantities :
  cell A,B,D,E,F,G,I,J
  edgex C,H
scalar : mu, tau
time : 500
computations :
  B[d1] = k0(tau,A)
  C[d2] = k1(B[nec])
  D[d1] = k2(C)
  E[d1] = k3(C)
  F[d1] = k4(D,C[nce])
  G[d1] = k0(mu,tau,E)
  H[d2] = k6(F)
  I[d1] = k7(G,H)
  J[d1] = k8(mu,I[ncc])
\end{filecontents*}

\begin{figure}[!h]
  \hspace{5mm}
  \begin{minipage}[!h]{0.98\textwidth}
    {\lstinputlisting[basicstyle=\small,mathescape,frame=single,language=Python,numbers=left]{exmsl.txt}}   
    \caption{Example of program using the Multi-Stencil Language. \label{fig:exmsl}}
  \end{minipage}
\end{figure}

\begin{enumerate}

\item The \texttt{mesh} keyword (Fig.\ref{fig:grammar},~l.1) introduces an identifier for $\mathcal{M}$, the single mesh of the simulation. For example \texttt{cart} in Fig.\ref{fig:exmsl},~l.1. Since the language is independent of the mesh topology, this is not used by the compiler but is syntactic sugar for the user.%and presents for reference only. 

\item The \texttt{mesh entities} keyword (Fig.\ref{fig:grammar},~l.2) introduces identifiers for the groups of entities $\mathcal{G}\in\Phi$. For example \texttt{cell} or \texttt{edgex} in Fig.\ref{fig:exmsl},~l.2. Again, this is syntactic sugar only as the compiler does not rely on the mesh topology.

\item The \texttt{computation domains} keyword (Fig.\ref{fig:grammar},~l.3) introduces identifiers for the computation domains $D\in\mathcal{D}$. For example \texttt{d1} and \texttt{d2} in Fig.\ref{fig:exmsl},~l.4-5. For reference, each domain is associated to a group of entities (Fig.\ref{fig:grammar},~l.12) such as \texttt{cell} for \texttt{d1} in Fig.\ref{fig:exmsl},~l.4.

\item The \texttt{independent} keyword (Fig.\ref{fig:grammar},~l.4) offers a way to declare that computation domains do not overlap, such as \texttt{d1} and \texttt{d2} in  Fig.\ref{fig:exmsl},~l.7. This is used by the compiler to compute dependencies between computations.

\item The \texttt{stencil shapes} keyword (Fig.\ref{fig:grammar},~l.5) introduces identifiers for $n\in\mathcal{N}$, the stencil shapes. For each shape, the source and destination group of entities (Fig.\ref{fig:grammar},~l.16) are specified. For example \texttt{nec} in Fig.\ref{fig:exmsl},~l.11 associates \texttt{cell} entities to each \texttt{edgex} entity.

\item The \texttt{quantities} keyword (Fig.\ref{fig:grammar},~l.6) introduces identifiers for $\delta\in\Delta$, the quantities with the group of entities they are mapped on (Fig.\ref{fig:grammar},~l.16). For example the quantities \texttt{C} and \texttt{H} are mapped on \texttt{edgex} entities.

\item The \texttt{scalar} keyword (Fig.\ref{fig:grammar},~l.7) introduces identifiers for $s\in\mathcal{S}$, the scalars. For example \texttt{mu} and \texttt{tau} in Fig.\ref{fig:exmsl},~l.15. 

\item Finally, the last part (Fig.\ref{fig:grammar},~l.8) introduces the computations loops. Each loop is made of two parts:
\begin{itemize}
\item the \texttt{time} keyword (Fig.\ref{fig:grammar},~l.22) introduces $conv$, the convergence criteria either as a number of iterations or as ${0,1}$ valued scalar (Fig.\ref{fig:grammar},~l.24). For example, 500 iterations are specified in Fig.\ref{fig:exmsl},~l.16,
\item the \texttt{computations} keyword (Fig.\ref{fig:grammar},~l.23) introduces identifiers for each computation $k\in\Gamma=(S,R,(w,D),comp)$. Each computation (Fig.\ref{fig:grammar},~l.26) specifies:
\begin{itemize}
 \item the quantity $w$ written and its domain $D$, for example in Fig.\ref{fig:exmsl},~l.22, kernel \texttt{k4} computes the quantity \texttt{F} on domain \texttt{d1},
 \item the read scalars $S$ and mesh quantities with their associated stencil shape $R$, for example in Fig.\ref{fig:exmsl},~l.16, \texttt{k4} reads \texttt{C} with the shape \texttt{nce} and \texttt{D} with the default \textit{identity} shape, it uses no scalar.
 \end{itemize}
\end{itemize}
\end{enumerate}

One can notice that in the example of Figure~\ref{fig:exmsl}, there are no (reduction) kernel associated to the scalars \texttt{mu} and \texttt{tau}.
In this case, those scalars are in fact constants.
One can also notice that the computation to execute for each kernel is not specified.
This is indeed not handled by MSL, which is mesh-agnostic, and which only generates a skeleton of the application (a parallel pattern of the application). Actually, a numerical computation is naturally mesh-dependent and is left to other languages. The numerical computation is specified elsewhere by referencing the identifier chosen here.

% To give a better idea of a multi-stencil program, a short example is given in Figure~\ref{fig:exmsl}. In this example, the mesh is called \textit{cart}. Two groups of mesh entities are defined \textit{cell} and \textit{edgex}. As previously explained, it is not needed in the grammar to give details on the topology of the mesh and the group of mesh entities. Two computation domains are given, \textit{d1} is a subpart of the entities of the group \textit{cell}, and \textit{d2} is a subpart of the entities of the group \textit{edgex}. It is declared that \textit{d1} and \textit{d2} are independant. Three stencil shapes are given. The second one for example \textit{nce} returns mesh entities of the group \textit{edgex} from an entity of the group \textit{cell}. Eight quantities are applied onto the group \textit{cell}, and two onto the group \textit{edgex}. Two scalars are defined, \textit{mu} and \textit{tau}. The time loop will iterate 500 times. Finally, nine computations are declared. For example, the computation \textit{k8} write the quantity \textit{J} onto the computation domain \textit{d1} by reading the scalar \textit{mu} and the quantity \textit{I} with a stencil shape \textit{n1}.

% The aim of the Multi-Stencil Language is to offer a way to describe a numerical simulation based on explicit numerical schemes. The language is mesh-agnostic as no information is asked to the user about the actual topology of the mesh. Moreover, information that is usefull for the implementation is also split from the MSL description. For this reason, the language grammar is simple. 

This section has introduced the grammar of MSL.
The next section will show how parallelization can be extracted from this simple language and how an empty parallel skeleton (or pattern) of the application can be generated. %, introducing implementation concerns in a second phase.
