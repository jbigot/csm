%%%%%%%%%%%%%%%%%%%%%%% file template.tex %%%%%%%%%%%%%%%%%%%%%%%%%
%
% This is a general template file for the LaTeX package SVJour3
% for Springer journals.          Springer Heidelberg 2010/09/16
%
% Copy it to a new file with a new name and use it as the basis
% for your article. Delete % signs as needed.
%
% This template includes a few options for different layouts and
% content for various journals. Please consult a previous issue of
% your journal as needed.
%
%%%%%%%%%%%%%%%%%%%%%%%%%%%%%%%%%%%%%%%%%%%%%%%%%%%%%%%%%%%%%%%%%%%
%
% First comes an example EPS file -- just ignore it and
% proceed on the \documentclass line
% your LaTeX will extract the file if required
%
\RequirePackage{fix-cm}
%
%\documentclass{svjour3}                     % onecolumn (standard format)
%\documentclass[smallcondensed]{svjour3}     % onecolumn (ditto)
%\documentclass[smallextended]{svjour3}       % onecolumn (second format)
\documentclass[twocolumn]{svjour3}          % twocolumn
%
\smartqed  % flush right qed marks, e.g. at end of proof
%
\usepackage{graphicx}
\usepackage{amsmath}
\usepackage{amssymb}
\usepackage{subfig}
\usepackage{caption}
\usepackage{cite}
\usepackage{filecontents}
\usepackage{listings}
\usepackage{tikz}
\usepackage[ruled,noend]{algorithm2e}
\usetikzlibrary{calc,arrows,shapes,automata,petri,positioning,decorations.markings,shadows}

\usepackage{todonotes}

%\newtheorem{theorem}{Theorem}[section]
\newtheorem{myprop}[theorem]{Proposition}
% \newenvironment{proof}[1][Proof]{\begin{trivlist}
% \item[\hskip \labelsep {\bfseries #1}]}{\end{trivlist}}
\newenvironment{mydef}[1][Definition]{\begin{trivlist}
\item[\hskip \labelsep {\bfseries #1}]}{\end{trivlist}}
\newenvironment{myth}[1][Theorem]{\begin{trivlist}
\item[\hskip \labelsep {\bfseries #1}]}{\end{trivlist}}

\def\pprec{\mathrel{\scalebox{.9}[1]{$\prec$}\mkern-3mu%
  \scalebox{.4}[1]{$\prec$}\mkern-5.5mu\scalebox{.4}[1]{$\prec$}}}
%
% \usepackage{mathptmx}      % use Times fonts if available on your TeX system
%
% insert here the call for the packages your document requires
%\usepackage{latexsym}
% etc.
%
% please place your own definitions here and don't use \def but
% \newcommand{}{}
%
% Insert the name of "your journal" with
% \journalname{myjournal}
%
\begin{document}

\title{The Multi-Stencil Language}
%\subtitle{Do you have a subtitle?\\ If so, write it here}

%\titlerunning{Short form of title}        % if too long for running head

\author{Helene Coullon         \and
        Chirtsian Perez \and
        Julien Bigot
}

%\authorrunning{Short form of author list} % if too long for running head

\institute{Helene Coullon \and Christian Perez \at
              INRIA \\
              46 Allée d'Italie\\
              69007 Lyon\\
              \email{helene.coullon@inria.fr}
           \and
           Julien Bigot \at
              CEA
}

%\date{Received: date / Accepted: date}
% The correct dates will be entered by the editor


\maketitle

\begin{abstract}
Insert your abstract here.
\keywords{First keyword \and Second keyword \and More}
\end{abstract}

%------------------------------------------------------------------------------
\section{Introduction}
\label{sect:introduction}
Contributions :
\begin{itemize}
\item Formalism of a multi-stencil program and its parallelization
\item The Multi-stencil Language: agnostic from the type of mesh and of the choosen implementation
\item A first implementation of the MSL compiler (static scheduling + fusion + dump to SkelGIS/components)
\item Evaluation of the different types of parallelism introduced + compiler + fusion
\end{itemize}
%------------------------------------------------------------------------------
\section{Computational model of Multi-Stencil Programs}
\label{sect:formalism}
To numerically solve a set of PDEs, iterative methods (finite difference, finite volume or finite element methods) are frequently used to approximate the solution through a discretized (step by step) phenomena. Thus, the continuous time and space domains are discretized so that a set of numerical computations are iteratively (time discretization) applied onto a mesh (space discretization). In other words, the PDEs are transformed to a set of numerical computations applied at each time step on all elements of the discretized space domain. Among those numerical computations is found a set of numerical schemes, also called \textit{stencil computations}, and a set of auxiliary computations also needed to perform the simulation, and also called \emph{local computations}.
This section gives formal definitions of a \textit{stencil program} and its computations. Then, the different parallelization techniques which can be applied on such program, are presented.

%-------------------------------------
\subsection{Time, mesh and data}

A mesh $\mathcal{M}$ defines the discretization of the continuous space domain $\Omega$ of a set of PDEs and is defined as followed. 

\begin{mydef}
\textit{A mesh is a connected undirected graph $\mathcal{M}=(V,E)$, where $V$ is the set of vertices and $E$ the set of edges. The set of edges $E$ of a mesh $\mathcal{M}=(V,E)$ does not contain bridges.}
\end{mydef}

\rmq{Pas très clair la suite; motivation? dom == some sub elements of M ?}
\begin{mydef}
$D_i$ is a set of elements of a mesh $\mathcal{M}=(V,E)$, constructed by a function $dom_i$ which defines a precise association between $V$ and $E$, $dom_i : V \times E \rightarrow D_i$.
\end{mydef}
For example, the set of cells $D_0$ in a Cartesian 2D mesh could be defined by exactly four vertices and four edges connected as a cycle. But we could also define another set of elements $D_1$ as the simple set of vertices $V$, etc.

A mesh can be structured (as Cartesian or curvilinear meshes), unstructured, regular or irregular (without the same topology for each element) and hybrid. 

\begin{figure}[!h]\begin{center}
  \resizebox{8cm}{!}{\includegraphics{./images/maillages.pdf}}
  \caption{From left to right, Cartesian, curvilinear and unstructured meshes.}
  \label{fig:mesh}
\end{center}\end{figure}

\begin{mydef}
The discretization of the continuous time domain $\mathcal{T}$ is denoted $T$ such that $\forall\mbox{ }t_i\mbox{, }t_{i+1} \in T\mbox{, }\exists\mbox{ }\Delta t \in \mathbb{R}$\mbox{, }$t_{i+1} = t_i + \Delta t$. Thus, $T$ is responsible for the iteration time steps of the numerical simulation. 
\end{mydef}

In a numerical simulation a set of data, or quantities, are applied onto the mesh and represent the set of values to compute, or to use, for computation.

\begin{mydef}
The set of data applied on the mesh is denoted by $\Delta$, such that $\delta \in \Delta$ is a function which associates each element $d \in D_i$ (the domain it is applied on) to a value $v \in V$, $\delta : D_i \rightarrow V$. In the rest of this paper, the domain of a data $\delta$ can be given by the function $domain(\delta)=D_i$.
\end{mydef}
One can notice that in applied mathematics, the signature of $\delta$ would be $\delta : D_i \times T \rightarrow V$, however when programming a numerical simulation it is not wise to store all values of each time iteration.

%-----------------------
\subsection{Computations}

\begin{mydef}
A numerical expression $\text{exp}$ is a function which represents how to compute a value for an element $d \in domain(w)=D_i$, using the set $R \subset \Delta$ of input data (read data), $\text{exp} : R \times D_i \rightarrow w \times D_i$.
\end{mydef}

\begin{mydef}
A computation $c$ of a numerical simulation is defined as $c(R,w,\text{exp})$, where $R \subset \Delta, w \in \Delta$ and $\text{exp}$ a numerical expression. %$D$ is one of the subsets $D_i \subset \mathcal{M}$, such that $w : D \rightarrow V$.
\end{mydef}
It has to be noticed that at each time iteration, all the elements of a mesh are computed. However, it happens that computations of the mesh elements are splitted in different domains, as for example the computation of the physical border. In this case additional $D_i$ can be specified for the mesh $\mathcal{M}$.

\begin{mydef}
The set of $n$ ordered computations of a numerical simulation is denoted $\Gamma = [c_i]_{0 \leq i \leq n-1}$, such that $\forall c_i,c_j \in \Gamma$, if $i \leq j$, then $c_i$ is computed before $c_j$, and $c_j$ can be computed only when $c_i$ is finished.
\end{mydef}

\begin{mydef}
A \textit{multi-stencil program} is defined by the quadruplet $\mathcal{MSP}(T,\mathcal{M},\Delta,\Gamma)$.
\end{mydef}
%If the number of computations in $\Gamma$ is $card(\Gamma)=n$, such that $\bigcup_{i=0}^{n-1}c_i = \Gamma$, then $\bigcup_{i=0}^{m-1}R_i \cup w_i \subseteq \Delta$.

As already mentioned, the ordered list $\Gamma$ can be composed of two different types of computations, stencil and local computations, which will be defined in the rest of this Section.

\begin{mydef}
The neighborhood $\mathcal{N}$ of an element $d \in D_i$ is a function to obtain a set of elements in any $D_k \subset \mathcal{M}$, $\mathcal{N} : D_i \rightarrow D_k \times D_k \times \dots$.
\end{mydef}
The function $\mathcal{N}$ is also sometimes called the \textit{stencil shape}, or the \textit{stencil} in applied mathematics. In this paper we distinguish a stencil shape from a \textit{stencil computation} defined as followed:

\begin{mydef}
A \textit{stencil computation} is defined as a quadruplet $s(R,w,\text{exp},\mathcal{N})$, where $R \subset \Delta$, and $w : D \rightarrow V \in \Delta$.
\end{mydef}
In a stencil computation $s$, $\forall d \in D$, the stencil numerical expression $\text{exp}$ is applied such that $w(d) = \text{exp}(R(d),R(\mathcal{N}(d))$. In this work, a stencil computation $s(R,w,\text{exp},\mathcal{N})$ always verifies $R \cap \{w\} = \emptyset$, otherwize an implicit numerical scheme has to be solve which is over the scope of this paper. As a result, the ordered list $\Gamma$ of a multi-stencil program can be composed of a set of stencil computations applied on one or more stencil shapes.

Figure~\ref{fig:ex} gives an example of a stencil computation $s(R,w,\text{exp},\mathcal{N})$, where $\mathcal{M}(V,E)$ is a two dimensional Cartesian mesh. A single domain $D=domain(w)$ is defined in this example and is composed of cells formed by a cycle of four vertices $v \in V$ and four edges $e \in E$. Furthermore, in this example $R=\{A\}$, $w=B$, and for $(x,y) \in D$ the neighborhood function is 
\begin{equation*}
\mathcal{N} : (x,y) \rightarrow \{(x,y+1),(x,y-1),(x+1,y),(x-1,y)\}.
\end{equation*}
Finally, the numerical expression of this example is 
\begin{equation*}
\text{exp}(A(x,y),A(\mathcal{N}(x,y)) = B(x,y) = A(x,y)+(A(x,y+1)+A(x,y-1)+A(x+1,y)+A(x-1,y))/4.
\end{equation*}

\begin{figure}[!h]\begin{center}
  \resizebox{8cm}{!}{\includegraphics{./images/example.pdf}}
  \caption{Example of a stencil computation.}
  \label{fig:ex}
\end{center}\end{figure}

The second type of numerical computation is a local computation.
\begin{mydef}
A local computation is a triplet $l(R,w,\text{exp})$, where $exp$ does not involve a neighborhood function $\mathcal{N}$.
\end{mydef}

A stencil program and stencil and local computations have been formally defined in this section. This formalism is used in the next Section to define two parallelization techniques of a multi-stencil program.



% Formalism :
% \begin{itemize}
% \item mesh / mesh entities and groups / computations domains
% \item data and scalars
% \item computations : stencil, local and reduction
% \item Multi-Stencil Program
% \end{itemize}
%------------------------------------------------------------------------------
\section{Parallelization of Multi-Stencil Programs}
\label{sect:parallelism}
\HC{has to be revised according to modifications}

% Multi-stencil mesh-based numerical simulation can be parallelized in various ways and is an interesting kind of application to take advantage of modern heterogeneous HPC architectures, mixing clusters, multi-cores CPUs, vectorization units, GPGPU and many-core accelerators.
As previously explained, in a computation $k(S,R,(w,D),comp)$, $comp$ is not handled by MSL. As a result, in the rest of this paper, and to simplify notations, we denote the same computation $k(S,R,(w,D))$.

%------------------------------
\subsection{Data parallelism}
\label{sect:dataparal}
In a data parallelization technique, the idea is to split quantities on which the program is computed into balanced sub-parts, one for each available resource. The same sequential program can afterwards be applied on each sub-part simultaneously, with some additioinal synchronizations between resources to update the data not computed locally, and thus to guarantee a correct result.

\medskip
More formally, the data parallelization of a multi-stencil program 
\begin{equation*}
\mathcal{MSP}(\mathcal{M},\Phi,\mathcal{D},\mathcal{N},\Delta, \mathcal{S},T,\Gamma)
\end{equation*}

consists in, first, a partitioning of the mesh $\mathcal{M}$ in $p$ balanced sub-meshes (for $p$ resources) $\mathcal{M}=\{\mathcal{M}_0,\dots,\mathcal{M}_{p-1}\}$. This step can be performed by an external graph partitionner~\cite{Pellegrini:1996:SSP:645560.658570,DBLP:conf/ieeehpcs/HeleneS13,lachat:hal-00768916} and is not adressed by this paper. 

As entities and quantities are mapped onto the mesh, the set of groups of mesh entities and the set of quantities $\Delta$ are partitionned the same way than the mesh: $\Phi=\{\Phi_0,\dots,\Phi_{p-1}\}$, $\Delta=\{\Delta_0,\dots,\Delta_{p-1}\}$. 

The second step of the parallelization is to identify in $\Gamma$ the needed synchronizations between resources to update data, and thus to build a new ordered list of computations $\Gamma_{sync}$.

\begin{mydef}
For $n$ the number of computations in $\Gamma$, and for $i,j$ such that $i<j<n$, a \textit{synchronization} is needed between $k_i$ and $k_j$, denoted $k_i \pprec k_j$, if $\exists (r_j,n_j) \in R_j$ such that $w_i=r_j$ and $n_j\neq identity$ ($k_j$ is a stencil computation). The quantity to synchronize is $\{w_i\}$.
\label{def:sync}
\end{mydef}

Actually, a synchronization is needed by the quantity read by a stencil computation (not local), if this quantity has been modified before, which means that it has been written before. This synchronization is needed because a neighborhood function $n \in \mathcal{N}$ of a stencil computation involves values computed on different resources.

However, as a multi-stencil program is an iterative program, computations which happen after $k_j$ at the time iteration $t$ have also been computed before $k_j$ at the previous time iteration $t-1$. For this reason another case of synchronization has to be defined.

\begin{mydef}
For $n$ the number of computations in $\Gamma$ and $j<n$, if $\exists (r_j,n_j) \in R_j$ such that $n_j\neq identity$ and for all $i<j$, $k_i \not \pprec k_j$, a \textit{synchronization} is needed between $k_l$ and $k_j$, where $j<l<n$, denoted $k_l \pprec k_j$, if $w_l=r_j$. The quantity to synchronize is $\{w_l\}$.
\label{def:sync2}
\end{mydef}

\begin{mydef}
A synchronization between two computations $k_i \pprec k_j$ is defined as a specific computation 
\begin{equation*}
k_{i,j}^{sync}(S,R,(w,D)), 
\end{equation*}
where $S=\emptyset$, $R=\{(r,n)\}=\{(w_i,n_j \in \mathcal{N}\}$, $(w,D)=(w_i,\bigcup_{\phi \in D_j} n_j(\phi)))$. In other words, $w_i$ has to be synchronized for the neighborhood $n_j$ for all entities of $D_j$.
\end{mydef}

\begin{mydef}
If $k_i \pprec k_j$, $k_j$ is replaced by the list
\begin{equation*}
[k_{i,j}^{sync}, k_j]
\end{equation*}
\end{mydef}

When data parallelism is applied, the other type of computation which is responsible for additional synchronizations is the reduction. Actually, the reduction is first applied locally on each subset of entities, on each resource. Thus, $p$ (number of resources) scalar values are obtained. For this reason, to perform the final reduction, a set of synchronizations are needed to get the final reduced scalar. As most parallelism libraries (MPI, OpenMP) already propose a reduction synchronization with its own optimizations, we simply choose to replace the reduction computation by itself anotated by $red$.

\begin{mydef}
A reduction kernel $k_j(S_j,R_j,(w_j,D_j))$, where $w$ is a scalar, is replaced by $k^{red}_j(S_j,R_j,(w_j,D_j))$. %if we denote by $w^r$, $0 \leq r<p$, the local scalar $w$ computed on the resource $r$, a reduction synchronization is defined as the specific computation 
% \begin{equation*}
% k_{j}^{sync}(S,R,(w,D),comp)
% \end{equation*}
\label{def:red}
\end{mydef}
% where, $S=\emptyset$, $R=\{(w^0,entity(w^0)) \dots (w^{p-1},entity(w^{p-1}))\}$, and $w=w_i$, $D=entity(w)=D_i$ and $comp=comp_i$.

One can notice that both types of synchronizations are performed by all resources.

\begin{mydef}
The concatenation of two ordered lists of respectively $n$ and $m$ computations $l_1=[k_i]_{0 \leq i \leq n-1}$ and $l_2=[k'_i]_{0 \leq i \leq m-1}$ is denoted $l_1 \cdot l_2$ and is equal to a new ordered list $l_3=[k_0,\dots,k_{n-1},k'_0,\dots,k'_{m-1}]$.
\end{mydef}

\begin{mydef}
From the ordered list of computation $\Gamma$, a new synchronized ordered list $\Gamma_{sync}$ is obtained from the call $\Gamma_{sync} = F_{sync}(\Gamma,0)$, where $F_{sync}$ is the recursive function defined in Algorithm~\ref{alg:sync}.
\end{mydef}

Algorithm~\ref{alg:sync} follows previous definitions to build a new ordered list which includes synchronizations. In this algorithm, lines 7 to 19 apply Definition~(\ref{def:sync}), lines 20 to 29 apply Definition~(\ref{def:sync2}), and finally lines 34 and 35 apply Definition~(\ref{def:red}). Finally, line 39 of the algorithm is the recursive call.

\begin{algorithm}
\caption{$F_{sync}$ recursive function}
\label{alg:sync}
\begin{algorithmic}[1]
\Procedure{$F_{sync}$} {$\Gamma$,$j$}
\State $k_j = \Gamma[j]$
\State $list = []$
\If {$j=|\Gamma|$}
\State return $list$
\ElsIf {$\exists (r_j,n_j) \in R_j$ such that $n_j\neq identity$}
\For {all $(r_j,n_j) \in R_j$ such that $n_j\neq identity$}
\State found = false
\For {$0 \leq i<j$}
\State $k_i = \Gamma[i]$
\If {$k_i \pprec k_j$}
\State found = true
\State $S = \emptyset$
\State $R = \{(w_i,n_j)\}$
\State $(w,D) = (w_i,\bigcup_{\phi \in D_j} n_j(\phi)))$
%\State $comp = identity$
\State $list.[k_{i;j}^{sync}(S,R,(w,D))]$%,comp)]$
\EndIf
\EndFor
\If {!found}
\For {$j<i\leq n$}
\State $k_i = \Gamma[i]$
\If {$k_i \pprec k_j$}
\State $S = \emptyset$
\State $R = \{(w_i,n_j)\}$
\State $(w,D) = (w_i,\bigcup_{\phi \in D_j} n_j(\phi)))$
%\State $comp = identity$
\State $list.[k_{i;j}^{sync}(S,R,(w,D))]$%,comp)]$
\EndIf
\EndFor
\EndIf
\State $list \cdot [k_j]$
\EndFor
\ElsIf {$w_j \in \mathcal{S}$}
\State $list.[k^{red}_j]$
\Else
\State $list.[k_j]$
\EndIf
\State return $list \cdot F_{sync}(\Gamma,j+1)$
\EndProcedure
\end{algorithmic}
\end{algorithm}


 The final step of this parallelization is to run $\Gamma_{sync}$ on each resource. Thus, for each resource $0 \leq r \leq p-1$ the multi-stencil program 
\begin{equation}
\mathcal{MSP}_r(\mathcal{M}_r,\Phi_r,\mathcal{D}_r,\mathcal{N},\Delta_r,\mathcal{S},T,\Gamma_{sync}),
\end{equation}
is performed.

\paragraph{\textbf{Example}} Figure~\ref{fig:exmsl} gives an example of a $\mathcal{MSP}$ program. From this example, the following ordered list of computation kernels can be extracted:
\begin{equation*}
\Gamma = [k_0,k_1,k_2,k_3,k_4,k_0,k_6,k_7,k_8]
\end{equation*}
From this ordered list of computation kernels $\Gamma$, and from the rest of the multi-stencil program, synchronizations can be automatically detected from the call to $F_{sync}(\Gamma,0)$ to get the synchronized ordered list of kernels:
\begin{equation}
\Gamma_{sync} = [k_0,k_{0;1}^{sync},k_1,k_2,k_3,k_{1;4}^{sync},k_4,k_0,k_6,k_7,k_{7;8}^{sync},k_8],
\label{eq:exsync}
\end{equation}
\begin{subequations}
where
\begin{align}
        k_{0;1}^{sync}=(\emptyset,\{(B,nce)\},(B,\cup_{\phi \in D_1} nce(\phi))),\\
        k_{1;4}^{sync}=(\emptyset,\{(C,nec)\},(C,\cup_{\phi \in D_4} nec(\phi))),\\
        k_{7;8}^{sync}=(\emptyset,\{(I,ncc)\},(I,\cup_{\phi \in D_8} ncc(\phi))).
\end{align}
\end{subequations}

%------------------------------
\subsection{Hybrid parallelism}
A task parallelization technique is a technique to transform a program as a dependency graph of different tasks. A dependency graph exhibits parallel tasks, or on the contrary sequential execution of tasks. Such a dependency graph can directly be given to a dynamic scheduler, or can statically be scheduled. In this paper, we introduce task parallelism by building the dependency graph between kernels of the sequential list $\Gamma_{sync}$. Thus, as $\Gamma_{sync}$ takes into account data parallelism, we introduce hybrid parallelism.

\begin{mydef}
For two computations $k_i$ and $k_j$, with $i < j$, it is said that $k_j$ is dependant from $k_i$ with a \emph{read after write} dependency, denoted $k_i \prec_{raw} k_j$, if $\exists (r_j,n_j) \in R_j$ such that $w_i=r_j$. In this case, $k_i$ has to be computed before $k_j$.
\end{mydef}

\begin{mydef}
For two computations $k_i$ and $k_j$, with $i < j$, it is said that $k_j$ is dependant from $k_i$ with a \emph{write after write} dependency, denoted $k_i \prec_{waw} k_j$, if $w_i = w_j$ and $D_i \cap D_j \neq \emptyset$. In this case, $k_i$ also has to be computed before $k_j$.
\end{mydef}

\begin{mydef}
For two computations $k_i$ and $k_j$, with $i < j$, it is said that $k_j$ is dependant from $k_i$ with a \emph{write after read} dependency, denoted $k_i \prec_{war} k_j$, if $\exists (r_i,n_i) \in R_i$ such that $w_j=r_i$. In this case, $k_i$ also has to be computed before $k_j$ is started so that values read by $k_i$ are relevant.
\end{mydef}

Those definitions are known as \emph{data hazards classification}. However, a specific condition on the computation domain, due to the specific domain of multi-stencils, is introduced for the write after write case.

\begin{mydef}
A directed acyclic graph (DAG) $G(V,A)$ is a graph where the edges are directed from a source to a destination vertex, and where, by following edges direction, no cycle can be found from a vertex $u$ to itself. A directed edge is called an arc, and for two vertices $v,u \in V$ an arc from $u$ to $v$ is denoted $(\overset{\frown}{u,v}) \in A$.
\end{mydef}

From an ordered list of computations $\Gamma_{sync}$, a directed dependency graph $\Gamma_{dep}(V,A)$ can be built finding all pairs of computations $k_i$ and $k_j$, with $i<j$, such that $k_i \prec_{raw} k_j$ or $k_i \prec_{waw} k_j$ or $k_i \prec_{war} k_j$. 

\begin{mydef}
For two directed graphs $G(V,A)$ and $G'(V',A')$, the union $(V,A)\cup (V',A')$ is defined as the union of each set $(V\cup V', A \cup A')$.
\end{mydef}

\begin{mydef}
From the synchronized ordered list of computation kernels $\Gamma_{sync}$, the dependency graph of the computations $\Gamma_{dep}(V,A)$ is obtained from the call $F_{dep}(\Gamma_{sync},0)$, where $F_{dep}$ is the recursive function defined in Algorithm~\ref{alg:dep}.

% \begin{equation*}
% F_{dep}(\Gamma_{sync},j) = 
% \begin{cases} 	\bullet (\{\},\{\}) \mbox{ if }j=|\Gamma_{sync}|\\
% 				\bullet (k_j, \{(\overset{\frown}{k_i,k_j})\mbox{, }\forall i < j \mbox{, } k_i\prec k_j \})\\
% 				\text{ } \qquad \cup F_{dep}(\Gamma_{sync},j+1) \mbox{ if }j<|\Gamma_{sync}|
% \end{cases}
% \end{equation*}
\end{mydef}

\begin{algorithm}
\caption{$F_{dep}$ recursive function}
\label{alg:dep}
\begin{algorithmic}[1]
\Procedure{$F_{dep}$} {$\Gamma_{sync}$,$j$}
\State $k_j = \Gamma_{sync}[j]$
\If {$j=|\Gamma_{sync}|$}
\State return $(\{\},\{\})$
\ElsIf {$j<|\Gamma_{sync}|$}
\State $G=(\{\},\{\})$
\For {$0 \leq i<j$}
\State $k_i = \Gamma_{sync}[i]$
\If {$k_i \prec_{raw} k_j$ or $k_i \prec_{waw} k_j$ or $k_i \prec_{war} k_j$}
\State $G = G \cup (k_j, \{(\overset{\frown}{k_i,k_j} \})$
\EndIf
\EndFor
\State return $G \cup F_{dep}(\Gamma_{sync},j+1)$
\EndIf
\EndProcedure
\end{algorithmic}
\end{algorithm}

This constructive function is possible because the input is an ordered list. Actually, if $k_i\prec k_j$ then $i<j$. As a result, $k_i$ is already in $V$ when the arc $(\overset{\frown}{k_i,k_j})$ is built.

One can notice that $\Gamma_{dep}$ is the dependency graph of the computations of a multi-stencil program, but it only takes into account a single time iteration. A complete dependency graph of the simulation could be built. This is a possible extension of this work.

\begin{myprop}
The directed graph $\Gamma_{dep}$ is an acyclic graph.
\end{myprop}

% \begin{proof}
% $\Gamma_{dep}$ is built from $\Gamma_{sync}$ which is an ordered and sequential list of computations. Moreover, each computation of the list $\Gamma_{sync}$ is associated to a vertex of $V$, even if the same computation is represented more than once in $\Gamma_{sync}$. As a result it is not possible to go back to a previous computation and to create a cycle.
% \end{proof}

As a result of the hybrid parallelization, each resource $0 \leq r \leq p-1$ perform a multi-stencil program, defined by
\begin{equation*}
\mathcal{MSP}_r(\mathcal{M}_r,\Phi_r,\mathcal{D}_r,\mathcal{N},\Delta_r,T,\Gamma_{dep}).
\end{equation*}
The set of computations $\Gamma_{dep}$ is a dependency graph between computation kernels $k_i$ of $\Gamma$ and synchronizations of kernels added into $\Gamma_{sync}$. $\Gamma_{dep}$ can be built from the call to 
\begin{equation*}
F_{dep}(F_{sync}(\Gamma,0),0).
\end{equation*}

\paragraph{\textbf{Example}} Figure~\ref{fig:exmsl} gives an example of $\mathcal{MSP}$ program. From $\Gamma_{sync}$ that has been built in Equation~(\ref{eq:exsync}), the dependency DAG can be built. For example, as $k_0$ computes $B$ and $k_1$ reads $B$, $k_0$ and $k_1$ becomes vertices of $\Gamma_{dep}$, and an arc $(\overset{\frown}{k_0,k_1})$ is added to $\Gamma_{dep}$. The overall $\Gamma_{dep}$ built from the call to $F_{dep}(\Gamma_{sync},0)$ is drawn in Figure~\ref{fig:depdep}.
\begin{figure}[h!]
\begin{center}
\begin{tikzpicture}[shorten >=1pt, node distance=2cm, on grid, auto]
   \node[] (c0) at (0,0) {$k_0$};
   \node[] (star1) at (1,0) {$k_{0;1}^{sync}$};
   \node[] (c1) at (2,0) {$k_1$};
   \node[] (c2) at (3,0.5) {$k_2$};
   \node[] (star4) at (3,1.5) {$k_{1;4}^{sync}$};
   \node[] (c3) at (3,-0.5) {$k_3$};
   \node[] (c4) at (4,0.5) {$k_4$};
   \node[] (c5) at (4,-0.5) {$k_5$};
   \node[] (c6) at (5,0.5) {$k_6$};
   \node[] (c7) at (6,0) {$k_7$};
   \node[] (star8) at (7,0) {$k_{7;8}^{sync}$};
   \node[] (c8) at (8,0) {$k_8$};
 
  \path[->]
    (c0) edge node {} (star1)
    (star1) edge node {} (c1)
    (c1) edge node {} (c2)
         edge node {} (c3)
         edge node {} (star4)
    (star4) edge node {} (c4)
    (c2) edge node {} (c4)
    (c4) edge node {} (c6)
    (c3) edge node {} (c5)
    (c5) edge node {} (c7)
    (c6) edge node {} (c7)
    (c7) edge node {} (star8)
    (star8) edge node {} (c8);
  \end{tikzpicture}
\caption{$\Gamma_{dep}$ of the example of program of Figure~\ref{fig:exmsl}}
\label{fig:depdep}
\end{center}
\end{figure}
% Parallelization formalism :
% \begin{itemize}
% \item Data parallelism (introduction of synchronizations)
% \item Task parallelism (computation of the dependency graph)
% \end{itemize}
%------------------------------------------------------------------------------
\section{The Multi-Stencil Language}
\label{sect:msl}
From the fomalism detailed in the previous section, the Multi-Stencil Language and its grammar can already be given. The reason why this grammar is sufficient to automatically extract parallelism will be explained in the next section.

\medskip
The grammar of the Multi-Stencil Language is given in Figure~\ref{fig:grammar}. A Multi-Stencil program is composed of eight parts. 

\begin{enumerate}
\item The first part represents the mesh of the simulation $\mathcal{M}$. As a single mesh is for now supported by the language, and as the topology of the mesh is not needed by the language compiler, \textit{meshid} (line 1) is a terminal symbol of the grammar. 
\item The second part (line 2) represents the set of groups of mesh entities needed by the simulation $\Phi$. It is defined as a list of groups $\mathcal{G}$. Again, as mesh entities are defined using the topology of the mesh, a group of mesh entities \textit{group} (defined line 13) is a terminal symbol of the grammar.
\item The third part (lines 3 and 4) represents the set of computation domains of the simulation $\mathcal{D}$. It is defined as a list of domains, each one defined (lines 14 and 15) as a subpart of a group of mesh entities, using the symbol "in".
\item The fourth part (lines 5 and 6) of the program is linked to the third one. It offers a way to declare that two computation domains are independants (lines 16 and 17). Two given computation domains $D_1$ and $D_2$ are independants if and only if $D_1 \cap D_2 = \emptyset$. It uses the symbol "and" to describe that $D_1$ and $D_2$ are independants.
\item The fith part (lines 7 and 8) represents the set of stencil shapes used by the simulation $\mathcal{N}$. A stencil shape (lines 18 and 19) is defined as a function "from" a group of mesh entities "to" another group of mesh entities.
\item The sixth part (line 9) represents the set of quantities of the simulation $\Delta$. It is composed of a list of quantities (lines 20 and 21) for which each line (line 22) begins with a group of mesh entities and a list of \textit{quanityid} which are applied onto the given group.
\item The seventh part (line 10) represents the set of scalars of the simulation $\mathcal{S}$. Each scalar (line 23) is a terminal symbol.
\item Finally, the eith part (line 11) represents the simulation computations. It is composed of a list of loop (line 24). Each \textit{loop} (line 25) is composed of a \textit{time} loop composed of \textit{iteration} (which represents $T$), and of a set of \textit{computations} (which represents $\Gamma$).
\begin{itemize}
\item A time loop, denoted \textit{iteration} in the grammar, can be a numerical constant value, \textit{num}, which directly indicates the number of time iterations, or a convergence \textit{criteria}. The \textit{criteria} symbol represents the function $conv:\mathcal{S}^n \rightarrow bool$ of the formalism. To define a criteria the terminal \textit{criteriaid} has to be set, and between bracket are indicated a list of scalars to read.
\item Each computation of the list of \textit{computations} follows the information of the formalism $k(S,R,(w,D),comp)$ except the $comp$ function which is an implementation concern. To be more easy to write the syntax is different though.
\end{itemize}
\end{enumerate}

\begin{filecontents*}{grammar.txt}
program ::= "mesh:" meshid 
            "mesh entities:" listgroup
            "computation domains:" 
                      listcompdom
            "independent:"
                      listinde
            "stencil shapes:"
                      liststencil
            "quantities:" listquantities
            "scalar" listscalar
            listloop

listgroup ::= group listgroup |  group
listcompdom ::= compdom listcompdom |  compdom
compdom ::= compdomid "in" group
listinde ::= inde listinde |  inde
inde ::= compdomid "and" compdomid
liststencil ::= stencil liststencil | stencil
stencil ::= stencilid "from" group "to" group
listquantities ::= data listquantities |  quantity
quantity ::= group listquantityid
listquantityid ::= quantityid listquantityid |  quantityid
listscalar ::= scalar listscalar | scalar
listloop ::= loop listloop | loop
loop ::=  "time:" iteration
          "computations:" listcomp
iteration ::= num | criteria
criteria ::= criteriaid "(" listscalarread ")"
listscalarread ::= scalar listscalarread |  scalar
listcomp ::= comp listcomp |  comp
comp ::= dataid "[" compdomid "]=" compid "({"listscalarread
                                          "},{"listdataread "})"
listdataread ::= dataread listdataread |  dataread
dataread ::= dataid "[" stencil "]" |  dataid
\end{filecontents*}

\begin{figure}[!h]
  \hspace{5mm}
  \begin{minipage}[!h]{0.98\textwidth}
    {\lstinputlisting[basicstyle=\small,mathescape,frame=single,language=Python,numbers=left]{grammar.txt}}   
    \caption{Grammar of the Multi-Stencil Language. \label{fig:grammar}}
  \end{minipage}
\end{figure}

\paragraph{\textbf{Example}} To give a better idea of a multi-stencil program, a short example is given in Figure~\ref{fig:exmsl}. In this example, the mesh is called \textit{cart}. Two groups of mesh entities are defined \textit{cell} and \textit{edgex}. As previously explained, it is not needed in the grammar to give details on the topology of the mesh and the group of mesh entities. Two computation domains are given, \textit{d1} is a subpart of the entities of the group \textit{cell}, and \textit{d2} is a subpart of the entities of the group \textit{edgex}. It is declared that \textit{d1} and \textit{d2} are independant. Three stencil shapes are given. The second one for example \textit{nce} returns mesh entities of the group \textit{edgex} from an entity of the group \textit{cell}. Eight quantities are applied onto the group \textit{cell}, and two onto the group \textit{edgex}. Two scalars are defined, \textit{mu} and \textit{tau}. The time loop will iterate 500 times. Finally, nine computations are declared. For example, the computation \textit{k8} write the quantity \textit{J} onto the computation domain \textit{d1} by reading the scalar \textit{mu} and the quantity \textit{I} with a stencil shape \textit{n1}.

\begin{filecontents*}{exmsl.txt}
mesh : cart
mesh entities : cell, edgex
computation domains :
  d1 in cell
  d2 in edgex
independent :
  d1 and d2
stencil shapes : 
  ncc from cell to cell
  nce from cell to edgex
  nec from edgex to cell
quantities :
  cell A,B,D,E,F,G,I,J
  edgex C,H
scalar : mu, tau
time : 500
computations :
  B[d1] = k0({tau},{A})
  C[d2] = k1({},{B[n2]})
  D[d1] = k2({},{C})
  E[d1] = k3({},{C})
  F[d1] = k4({},{D,C[n3]})
  G[d1] = k0({mu,tau},{E})
  H[d2] = k6({},{F})
  I[d1] = k7({},{G,H})
  J[d1] = k8({mu},{I[n1]})
\end{filecontents*}

\begin{figure}[!h]
  \hspace{5mm}
  \begin{minipage}[!h]{0.98\textwidth}
    {\lstinputlisting[basicstyle=\small,mathescape,frame=single,language=Python,numbers=left]{exmsl.txt}}   
    \caption{Example of program using the Multi-Stencil Language. \label{fig:exmsl}}
  \end{minipage}
\end{figure}

The aim of the Multi-Stencil Language is to offer a way to describe a numerical simulation based on explicit numerical schemes. The language is mesh-agnostic as no information is asked to the user about the actual topology of the mesh. Moreover, information that is usefull for the implementation is also split from the MSL description. For this reason, the language grammar is simple. 

In the next section will be shown how parallelization can be extracted from this simple language and how an empty parallel skeleton of the application can be generated, introducing implementation concerns in a second phase.
% The parallel empty skeleton of the application can be computed automatically from the language :
% \begin{itemize}
% \item grammar
% \item example from the language to the dependency graph
% \end{itemize}
%------------------------------------------------------------------------------
\section{Static Scheduling}
\label{sect:msp}
In this section we detail a static scheduling of $\Gamma_{dep}$ by using minimal series-parallel directed acyclic graphs. Such a static scheduling may not be the most efficient one, but it offers a simple fork/join task model which make possible the design of a performance model. Moreover, such a scheduling offers a simple way to propose a fusion optimization. 

%--------------------
\subsection{Series-Parallel graph}

In 1982, Valdes \& Al~\cite{Valdes:1979:RSP:800135.804393} have defined the class of Minimal Series-Parallel DAGs (MSPD). Such a graph can be decomposed as a serie-parallel tree, denoted $TSP$, where each leaf is a vertex of the MSPD DAG it represents, and whose internal nodes are labelled $S$ or $P$ to indicate respectively the series or parallel composition of sub-trees. Such a tree can be considered as a fork-join model and as a static scheduling.

Valdes \& Al~\cite{Valdes:1979:RSP:800135.804393} have identified a forbidden shape, or subgraph, called $N$, such that the following property is verified :

\begin{myth}
The transitive reduction of a DAG $G$ is MSPD if and only if it does not contain $N$ as an induced subgraph.
\end{myth}

To remove the forbidden N-shapes from the transitive reduction of $\Gamma_{dep}=(V,E)$, we have chosen to apply an over-constraint with the relation $k_0 \prec k_3$, such that a complete bipartite graph is created for the sub-dag, and can be translated to a series-parallel decomposition, as illustrated in Figure~\ref{fig:allover}.

\begin{figure}[h!]
\begin{center}
\begin{tikzpicture}[shorten >=1pt, node distance=2cm, on grid, auto]
   \node[] (c0) at (0,0) {$k_0$};
   \node[] (bc0) at (-1,0) {};
   \node[] (c1) at (2,0) {$k_1$};
   \node[] (c2) at (0,-1) {$k_2$};
   \node[] (bc2) at (-1,-1) {};
   \node[] (c3) at (2,-1) {$k_3$};
 
  \path[->]
    (bc0) edge [dotted] node {} (c0)
    (bc2) edge [dotted] node {} (c2)
    (c0) edge node {} (c1)
          edge [dashed] node [swap] {} (c3)
    (c2) edge node {} (c1)
        edge node {} (c3);
  \end{tikzpicture}
\caption{Over-constraint on the forbidden $N$ shape.}
\label{fig:over}
\end{center}
\end{figure}

After these over-constraints are applied, $\Gamma_{dep}$ is MSPD. Valdes \& Al~\cite{Valdes:1979:RSP:800135.804393} have proposed a linear algorithm to know if a DAG is MSPD and, if it is, to decompose it to its associated binary decomposition tree. As a result, the binary tree decomposition algorithm of Valdes \& Al can be applied on $\Gamma_{dep}$ to get the $TSP$ static scheduling of the multi-stencil program.

\paragraph{\textbf{Example}} The Serie-Parallel tree decomposition of the example given in Figure~\ref{fig:exmsl}, which is built from the dependency graph of Figure~\ref{fig:depdep} is given in Figure~\ref{fig:tree}.

\begin{figure}[h!]
\begin{center}
\begin{tikzpicture}[shorten >=1pt, node distance=2cm, on grid, auto]
   %\node[circle,draw=black,fill=black,scale=0.3] (c0s) at (0,0) {};
   %\node[circle,draw=black,fill=black,scale=0.3] (c8d) at (9,0) {};

   \node[] (s8) at (4.5,0.5) {$\mathcal{S}$};
   \node[] (s9) at (3,1.5) {$\mathcal{S}$};

   %reduction c0 c1
   \node[] (s0) at (1.5,2.5) {$\mathcal{S}$};
   \node[] (s1) at (1,3.5) {$\mathcal{S}$};
   \node[] (c0) at (0.5,4.5) {$k_0$};
   \node[] (star1) at (1.5,4.5) {$k_{0;1}^{sync}$};
   \node[] (c1) at (2,3.5) {$k_1$};
   
   %reduction c7 c8
   \node[] (s2) at (7.5,1.5) {$\mathcal{S}$};
   \node[] (s3) at (7,2.5) {$\mathcal{S}$};
   \node[] (c7) at (6.5,3.5) {$k_7$};
   \node[] (star8) at (7.5,3.5) {$k_{7;8}^{sync}$};
   \node[] (c8) at (8,2.5) {$k_8$};

   \node[] (p1) at (4.5,2.5) {$\mathcal{P}$};
   %reduction c3 c5
   \node[] (s5) at (5.5,3.5) {$\mathcal{S}$};
   \node[] (c3) at (5,4.5) {$k_3$};
   \node[] (c5) at (6,4.5) {$k_0$};
   %reduction c2 c6
   \node[] (s6) at (3.5,3.5) {$\mathcal{S}$};
   \node[] (p0) at (2.8,4.5) {$\mathcal{P}$};
   \node[] (c2) at (2.3,5.5) {$k_2$};
   \node[] (star4) at (3.3,5.5) {$k_{1;4}^{sync}$};
   \node[] (s7) at (4.2,4.5) {$\mathcal{S}$};
   \node[] (c4) at (3.8,5.5) {$k_4$};
   \node[] (c6) at (4.8,5.5) {$k_6$};
 
  \path[->]
    %(c0s) edge node {} (c8d)

    (s8) edge node {} (s9)
         edge node {} (s2)
    (s9) edge node {} (p1)
         edge node {} (s0)

    %reduction c0 c1
    (s0) edge node {} (s1)
         edge node {} (c1)
    (s1) edge node {} (c0)
         edge node {} (star1)
    %reduction c7 c8
    (s2) edge node {} (s3)
         edge node {} (c8)
    (s3) edge node {} (c7)
        edge node {} (star8)

    (p1) edge node {} (s6)
         edge node {} (s5)
    %reduction c3 c5
    (s5) edge node {} (c3)
         edge node {} (c5)
    %reduction c2 c6
    (s6) edge node {} (p0)
         edge node {} (s7)
    %reduction c2 *4
    (p0) edge node {} (c2)
         edge node {} (star4)
    (s7) edge node {} (c4)
         edge node {} (c6);
  \end{tikzpicture}
\caption{Serie-Parallel tree decomposition of the example of program of Figure~\ref{fig:exmsl}}
\label{fig:tree}
\end{center}
\end{figure}

%--------------------
\subsection{Performance model}

%--------------------
\subsection{Fusion optimization}

Using MSL, it is possible to ask for data parallelization of the application, or for an hybrid parallelization. Even though the MSL language is not dedicated to produce very optimized stencil codes, but to produces the parallel pattern of the application, building the $TSP$ tree make available an easy optimization when the data parallelization technique is the only one used. This optimization consists in proposing a valid merge of some computation kernels inside a single space loop. As a result, the user can use this valid fusion of kernels or not when implementing those. 

Those fusions can be computed from the canonical form of the $TSP$ tree decomposition. The canonical form consists in recursively merging successive $S$ vertices or successive $P$ vertices of $TSP$.

The fusion function $F_{fus}$ is described in Algorithm~\ref{alg:fus}, where $parent(k)$ returns the parent vertex of $k$ in the tree, and where $k_{i;j}^{fus}$ represents the fusion of $k_i$ and $k_j$ keeping the sequential order $i;j$ if $i$ is computed before $j$ in $TSP$. Finally, $type(k)$ returns $comp$ if the kernel is a computation kernel, and $sync$ otherwise.

\begin{algorithm}
\caption{$F_{fus}$}
\label{alg:fus}
\begin{algorithmic}[1]
\Procedure{$F_{fus}$} {$TSP(V,E)$}
\For {$(k_i,k_j) \in V^2$}
\If {parent($k_i$)==parent($k_j$)}
\If {$type(k_i)=type(k_j)=comp$}
\If {parent($k_i$)==$S$}
\If {$D_i==D_j$}
\State propose the fusion $k_{i;j}^{fus}$
\EndIf
\ElsIf {parent($k_i$)==$P$}
\If {$D_i==D_j$ and $R_i \cap R_j \neq \emptyset$}
\State propose the fusion $k_{i;j}^{fus}$
\EndIf
\EndIf
\EndIf
\EndIf
\EndFor
\EndProcedure
\end{algorithmic}
\end{algorithm}

We are not arguing that such a simple fusion algorithm could be as good as complex cache optimization techniques which can be found in stencil DSLs for example~\cite{spaaTangCKLL11}. However, this fusion takes place at a different level and can bring performance improvments as it will be illustrated in Section~\ref{sect:eval}. This fusion algorithm relies on very simple statements:
\begin{itemize}
\item Two successive computation kernels $k_i$ and $k_j$ which are under the same parent vertex $S$ in TSP are, by construction, data dependant. As a result, what is written by the first one is read by the second one. Thus, at least one data is common to those computations (the one written by $k_i$). Thus, if the computation domains verify $D_i=D_j$, the fusion of $k_i$ and $k_j$ will decrease cache misses.
\item Two successive computation kernels $k_i$ and $k_j$ which are under the same parent vertex $P$ in TSP are not, by construction, data dependant. However, if the computation domains verify $D_i=D_j$, and if $R_i \cap R_j \neq \emptyset$ cache misses could also be decreased by the fusion $k_{i;j}^{fus}$.
\end{itemize}







% From the dependency graph we study a first dump: static scheduling
% \begin{itemize}
% \item description of the algorithm to get the binary tree decomposition of a minimal serie-parallel graph
% \item use of the serie-parallel tree decomposition to detect fusions of kernels
% \item example (the same one) from the dependency graph to the binary tree
% \end{itemize}
%------------------------------------------------------------------------------
\section{Evaluation}
\label{sect:eval}
\begin{itemize}
\item short description of the implementation (compiler in Python, DDS = SkelGIS, Components)
\item evaluation of the compiler (execution times of the different steps)
\item evaluation of the data parallelism (weak/strong scaling)

\begin{figure}[!h]\begin{center}
  \resizebox{8cm}{!}{\includegraphics{../results/weak_scaling/400/median_weak.pdf}}
  \caption{weak-scaling 400x400 blocks.}
  \label{fig:mesh}
\end{center}\end{figure}

\begin{figure}[!h]\begin{center}
  \resizebox{8cm}{!}{\includegraphics{../results/weak_scaling/800/median_weak.pdf}}
  \caption{weak-scaling 800x800 blocks}
  \label{fig:mesh}
\end{center}\end{figure}

\begin{figure}[!h]\begin{center}
  \resizebox{8cm}{!}{\includegraphics{../results/strong_scaling/10K_1K/median_strong.pdf}}
  \caption{strong-scaling 10kx10k, 1k iterations}
  \label{fig:mesh}
\end{center}\end{figure}

\item evaluation of the fusion

\begin{figure}[!h]\begin{center}
  \resizebox{8cm}{!}{\includegraphics{../results/task_scaling/500_200/fusVSbase.pdf}}
  \caption{strong scaling 500x500, 200 iterations, with and without fusion}
  \label{fig:mesh}
\end{center}\end{figure}

\item evaluation of the hybrid parallelism

\begin{figure}[!h]\begin{center}
  \resizebox{8cm}{!}{\includegraphics{../results/task_scaling/500_200/base_close_median.pdf}}
  \caption{MPI only vs MPI+threads with \emph{close} scheduling policy (OpenMP)}
  \label{fig:mesh}
\end{center}\end{figure}

\begin{figure}[!h]\begin{center}
  \resizebox{8cm}{!}{\includegraphics{../results/task_scaling/500_200/base_spread_median.pdf}}
  \caption{MPI only vs MPI+threads with \emph{spread} scheduling policy (OpenMP)}
  \label{fig:mesh}
\end{center}\end{figure}

\item analytic model for the results

\begin{figure}[!h]\begin{center}
  \resizebox{8cm}{!}{\includegraphics{../results/task_scaling/500_200/analytic/times.pdf}}
  \caption{Computation vs communication times in the MPI only application}
  \label{fig:mesh}
\end{center}\end{figure}

\end{itemize}
%------------------------------------------------------------------------------
\section{Related work}
\label{sect:rel}
\begin{itemize}
\item Lizst, Pochoir, PATUS for optimizations/parallelization of a single stencil kernel + mesh dependent
\item Halide for pipeline of stencil computations = mesh dependent
\item Reuse of external solutions as SkelGIS, GA, Lizst, Pochoir etc.
\end{itemize}
%------------------------------------------------------------------------------
\section{Conclusion}
\label{sect:concl}
%------------------------------------------------------------------------------

%\begin{acknowledgements}
%If you'd like to thank anyone, place your comments here
%and remove the percent signs.
%\end{acknowledgements}

\bibliographystyle{spbasic}
\bibliography{biblio}

\end{document}
% end of file template.tex

