%--------------------
%The multi-stencil language is a descriptive language. MSL follows a light grammar which extracts information needed to build an empty parallel pattern of a numerical simulation (a multi-stencil program), also called an empty parallel skeleton of the simulation. From the language grammar are automatically detected where synchronizations are needed in order to apply a data parallelization technique, and how computations are dependent from each other to apply a task parallelization technique.

%The multi-stencil language splits the simulation description from its implementation concerns. For this reason, the multi-stencil language is mesh-agnostic, which means that no information on the actual topology of the mesh is given by the user. For this reason too, no numerical code is asked to the user. Those implementation concerns are deported to a second phase where other languages or libraries can be used. This phase is called the \emph{dump} in the overall compiler algorithm.

The MSL compiler takes a file written in the grammar described in Section~\ref{sect:msl} as input and generates an application skeleton.
In this skeleton, the user still has to fill the functions corresponding to the various computation kernels.
The overall behavior of the compiler is as follow: %presented in Algorithm~\ref{alg:compiler}.
\begin{enumerate}
 \item it parses the input file and generates $\Gamma$, the list of computation kernels,
 \item from $\Gamma$, it builds $\Gamma_{sync}$, the list including synchronizations for data parallelism using Algorithm $F_{sync}$ introduced in Section~\ref{sect:parallelism},
 \item from $\Gamma_{sync}$, it builds $\Gamma_{dep}$, the DAG supporting hybrid parallelism using Algorithm $F_{dep}$ introduced in Section~\ref{sect:parallelism},
 \item it then removes the N-Shapes from $\Gamma_{dep}$ to get a MSPD graph, and generates its serie-parallel binary tree decomposition $TSP$,
 \item it performs the fusion of kernels in $TSP$ if required,
 \item finally it dumps an application structure matching $TSP$.
% Implementation choices of the dump are described in the next subsection.
\end{enumerate}

%JB: un enumerate me semble suffisant
% \begin{algorithm}
% \caption{MSL Compiler}
% \label{alg:compiler}
% \begin{algorithmic}[1]
% \Procedure{compileMSL} {file}
% \State $\Gamma$ = parser(file)
% \State $\Gamma_{sync} = F_{sync}(\Gamma,0)$
% \State $\Gamma_{dep} = F_{dep}(\Gamma_{sync},0)$
% \State removeNSH($\Gamma_{dep}$)
% \State $TSP = F_{tsp}(TSP,root(TSP))$
% \If {data parallel}
% \State $TSP = F_{fus}(TSP)$
% \EndIf
% \State dump(TSP)
% \EndProcedure
% \end{algorithmic}
% \end{algorithm}

%--------------------
% \subsection{Implementation choices}

The abstract definition of MSL enables to use a very wide range of languages for the dump.
In this paper, we use a compiler that targets the Low Level Components~\cite{l2c} (\llc), a component model for C++ supporting high-performance applications.
Similar to classes in object-oriented models, components specify the services they provide, but in addition they also specify the services they require.
This enables to build applications by assembling components in a second phase through the matching of requirements of some components with the services provided by others.
In the case of MSL, this makes it possible to clearly separate in distinct components the features that are implemented:
\begin{itemize}
 \item by reusing the same code every-time such as for the time iteration logic,
 \item by generating some code that depends on $TSP$ and thus from the input file such as for the task parallelism logic,
 \item or by letting the user fill-in an empty skeleton component such as for the computation kernels.
\end{itemize}
By enforcing the use of well defined interfaces in the generated code, \llc makes it easier to replace an implementation by another to reconsider an implementation choice.
It also simplifies the potential addition of new features to MSL or the coupling of MSL with another domain-specific language for example.

The two main other choices in the code generation concern the technologies used for the data and task parallelizations.
For the data-parallelization, we rely on SkelGIS, a C++ embedded DSL~\cite{CPE:CPE3494} which offers distributed data structures (meshes) for numerical simulations, and programming interfaces to easily write codes using them.
SkelGIS is implemented over MPI~\cite{Graham2009MSE} and can therefore be used on distributed memory architectures such as clusters.
Two kind of meshes are offered by SkelGIS: \textbf{a)} a distributed two dimensional Cartesian mesh~\cite{DBLP:conf/ieeehpcs/HeleneS13}, and \textbf{b)} a distributed graph of Cartesian meshes (hybrid mesh)~\cite{DBLP:conf/europar/CoullonL14}.
The use of different back-ends for data-parallelism such as for example the distributed unstructured mesh proposed by PaMPA~\cite{lachat:hal-00768916} will be the subject of future work.

For the task parallelism, we use OpenMP~\cite{660313}.
OpenMP targets shared-memory platforms only, but the degree of task-parallelism is typically much lower than that of data parallelism.
With the increasing number of cores of current machines this has not proved to be a problem in our experience.
While the version 4 of OpenMP has introduced explicit support for tasks, our implementation only requires version 3 whose fork-join model is well suited for the static scheduling introduced in Section~\ref{sect:msp}.
The use of dynamic schedulers such as provided by libgomp\footnote{\url{https://gcc.gnu.org/projects/gomp/}}, StarPU~\cite{Augonnet2011}, or XKaapi~\cite{Gautier:2013:XRS:2510661.2511383} to directly execute the DAG will be the subject of some future work.

As a result, the MSL compiler generates an hybrid code which uses both SkelGIS and OpenMP.
It also generates empty components where the user must provide local sequential implementations of the kernels using well defined interfaces of the component to iterate over the local data subdomain.
The following section will evaluate this compiler as well as the code it generates.
