Many domain specific languages have been proposed for stencil computations. Each one has its own specificities and answers to a specific stencil case or to a specific additionnal optimization. For example, Pochoir~\cite{spaaTangCKLL11} works on cache optimization techniques for stencils applied onto Cartesian meshes. On the other hand, PATUS~\cite{citeulike12258902} proposes to add a parallelization strategy grammar to its stencil language to perform an auto-tuning parallelization strategy. Moreover, Halide~\cite{Ragan-Kelley:2013:HLC:2491956.2462176}  proposes an optimization and parallelization of a pipeline of stencil codes, while ExaSlang~\cite{Schmitt:2014:EDL:2691166.2691171} is specific to multigrid numerical methods etc.

The reason why each of those languages are implemented from scratch to answer its own particularities and its own optimizations is because each language is built and thought as a single block, which makes impossible or difficult code reuse from one language to another. In other words domain specific languages (for stencils or other domains) suffer from a lack of software engineering properties, which could increase the productivity to build a new language, by code-reuse, and also the maintainability of languages, with more separation of concerns. As far as we know, a single work proposes a framework to create DSL, which afterwards will be easier to compose and combine together~\cite{Sujeeth:2013:CRC:2524984.2524988}, which is interesting for inter-domains applications. However, the point argued by this paper, and by the Multi-Stencil Language (MSL), is that DSL conception must be studied to maximize its usefulness for different types of applications, while keeping a good performance. This maximization is directly linked to the abstraction level proposed to the end-user, and also to separation of concerns and code reuse improvements.

MSL is itself a domain specific language for stencil-based simulations. It offers a way to give a mesh-agnostic description of a simulation, which could be common to different cases of simulations, and thus facilitates reuse of existing underlying data structures, optimizations or parallelization techniques. In other words, MSL extracts where parallelization and optimizations are needed into the simulation, by producing an empty parallel pattern, but not how to do it, which is left to existing languages (SkelGIS and OpenMP in this paper), and which is an implementation concern.

Liszt~\cite{DeVito:2011:LDS:2063384.2063396} and Nabla~\cite{Camier:2015:IPP:2820083.2820107}, both offer languages for stencils applied onto any kind of mesh, from Cartesian to unstructured meshes. The mesh which is needed into the simulation can be built from a set of available symbols in the grammar of each language. Thus, those languages generalize the definition of a mesh, as it is proposed into the MSL formalism. 
However, two main differences can be noticed. 

First, the formalism is more flexible in MSL. For example, a computation can be applied onto a subset of the space domain which is not possible with Liszt and Nabla. This functionnality is important in numerical simulations yet. For example, many computations performed in the simulation studied in this paper have to be applied onto a subpart of the overall space domain.

Second, in Liszt and Nabla the description of a simulation is not splitted from its implementation concerns. In other words, the topology of a mesh as well as numerical codes are given at the same time than the description of the different computations to apply into the simulation, while MSL splits those two different phases. Thus, MSL improves separation of concerns. Actually, a numerician could describes the different computations to perform into the simulation, while another one, later, could focuss on the implementation and the numerical code. This also facilitates reuse of other languages. 

Finally, the parallelization techniques proposed in this paper, take place at paradigm level, with data parallelism and hybrid (data plus task) parallelism. Thus the parallel pattern of the simulation does not need details onto parallel architectures (distributed or shared memories, with or without accelerators). This makes possible a large panel of backend architectures and languages. In other words MSL improves portability.