The solution presented in this paper is a domain specific language, to describe a multi-stencil program (a numerical simulation), and a compiler which transforms this description to a parallel component-based application of the overall simulation.

The MS language (MSL) is an agnostic descriptive language for multi-stencil simulations. Agnostic means that the description of a numerical simulation, using MSL, does not need any knowledge on the mesh type, the stencil shapes, or the final implementation constraints. Descriptive, on the other hand, means that the terminal states of the language grammar are string identifiers and that the language does not handle description of numerical computations, but only description of a multi-stencil simulation, in a coarse-grain fashion.

As illustrated in Figure~\ref{fig:mscac}, the MS language compiler, called MSCAC for \emph{Multi-Stencil Component Assembly Compiler}, is composed of two different parts. First, the computation compiler, called \emph{MSC}, is responsible for the parallelization of computations and for its dump to a component assembly; second, the data compiler, called \emph{MSD}, handles a distributed data structure, data mapped onto it, and their relations with computations, as a final overall component assembly.

\begin{figure}[h!]
\begin{center}
\begin{tikzpicture}[remember picture,
  inner/.style={rectangle,rounded corners=3pt,thick,inner sep=5pt},
  outer/.style={rectangle,rounded corners=3pt,thick,inner sep=5pt}
  ]
  \node (MSL) {MS Language};
  \node[outer,below=of MSL,draw=black] (MSCAC) {
    \begin{tikzpicture}
      \node [inner,draw=black,thick] (MSC)  {MSC};
      \node [inner,right=of MSC,draw=black,thick,,fill=gray!30!white,dotted] (MSD)  {MSD};
    \end{tikzpicture}
  };
  \node [below=of MSCAC] (CA) {Component Assembly};
  \node [left=0.1 of MSCAC] (CAid) {MSCAC};
  \draw[->] (MSL) -- (MSCAC);
  \draw[->] (MSCAC) -- (CA);
\end{tikzpicture}
\caption{MS Language and MSCAC compiler}
\label{fig:mscac}
\end{center}
\end{figure}

This paper presents the MS language and the MSC part of the compiler. The MSC compiler produces a static scheduling of the set of computations, ready to dump to a parallel language, such as OpenMP~\cite{660313} or HPF~\cite{219857}, but also ready to dump to a component assembly which is the main contribution of this work.

\begin{figure*}
\begin{center}
\begin{tikzpicture}[shorten >=1pt, node distance=2cm, on grid, auto]
   \node[component] (D) at (0,0) {Driver};
   \node[provide] (Dp) at (-1,0) {};
   \node (Ds) at (-1.5,0) {start};
   \node[use] (Du1) at (1,0.5) {};
   \node[use] (Du2) at (1,-0.5) {};

   \node[component,right=1cm of Du2] (DA) {DriverApp};
   \node[use] (DAu1) at (3.5,0) {$m$};
   \node[use] (DAu2) at (3.5,-1) {};

   \node[component] (T) at (4.5,-1) {Time};
   \node[use,right=1cm of T] (Tu) {};
   \node[component,right=2cm of Tu] (C) {Computations};
   \node[use,right=2cm of C] (Cu) {};
   \node[below=0.2cm of Cu] (star) {$*$};

   \node[component,right=8cm of DAu1] (Data) {Data};
   \node[use,right=1cm of Data] (Datau) {};

   \node[component,right=13cm of Du1] (DDS) {DDS};
 
  \path[-]
    (Dp) edge node {} (D)
    (D.east) edge node {} (Du1)
      edge node {} (Du2)
    (Du1) edge node {} (DDS.west)
    (Du2) edge node {} (DA)
    (DA.east) edge node {} (DAu1)
      edge node {} (DAu2)
    (DAu1) edge node {} (Data.west)
    (DAu2) edge node {} (T)
    (T) edge node {} (Tu)
    (Tu) edge node {} (C)
    (C) edge node {} (Cu)
    (Cu) edge node {} (Data.west)
    (Data) edge node {} (Datau)
    (Datau) edge node {} (DDS.west);
\end{tikzpicture}
\caption{Shape of the generated component assembly.}
\label{fig:assembly}
\end{center}
\end{figure*}

It has to be noticed that in this paper, the MSD compiler is a first adhoc version which works for a precise distributed data structure, the SkelGIS library~\cite{HeleneLS13,HeleneLS14,HeleneEuroPar14,CPE:CPE3494}. However, using the system of python string templates~\footnote{\url{https://docs.python.org/2/library/string.html}} to define the component assembly part which manages data, it is already possible to use another distributed data structure, as for example Global Arrays~\cite{Nieplocha:2006:AAP:1125980.1125985}. This part of the compiler takes place in a larger research project to propose domain decomposition skeletons using component models.

The overall compiler MSCAC produces a ready-to-fill component-based parallel structure of the simulation. In this final component assembly, it is needed to write computation components, using interfaces of the distributed data structure choosen by MSD. Thus, this work is complementary to stencil compilers or to distributed data structure solutions.