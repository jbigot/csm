Programming complex scientific applications on high performance architectures, such as clusters of multi-core nodes containing accelerators, requires for a scientist to be an expert in his own domain, but also in low level parallel and high performance programming. It can take months to years for a scientist to learn HPC programming, as it is not his main activity. As hardware continue to evolve to become more powerful but also usually more complex and difficult to use efficiently, the learning is never over.

Domain Specific Languages (DSLs) are a promising solution to hide intricate details of HPC programming from non experts.
By being specific to a well-defined set of problems, those languages enable the generation of well parallelized and optimized code for various architectures while requiring only a limited amount of information from the developer; part of the knowledge is directly embedded in the compiler.
A critical question however concerns the specificity of DSLs.
Actually, the less specific the DSL is, the more information its users must provide, and the less automatic optimizations can be embedded in the compiler.
%On the other side of the scope
Conversely, the more specific the DSL is, the less applications can use it, and the more difficult it is to amortize the implementation cost of the DSL (language, compiler, and runtime).
Programming complexity, as well as non-maintainability and non-portability often seem to be simply transferred from scientists to DSL designers.
% Moreover, one can consider that end users need as much DSLs as domains, which keeps the amount of work to write and maintain up-to-date DSLs out of hands.

For this reason, solutions to ease the design and implementation of DSLs have recently appeared~\cite{Fernandez:2014:DFL:2691166.2691168}. The main idea of those solutions is to propose a single DSL or a framework to write new DSLs. In this paper, we study a first contribution to another approach to improve maintainability and portability in DSLs. This approach proposes a transformation from a DSL to a component-based runtime. Component-based Software Engineering has proved many times good properties for code re-use, separation of concerns, maintainability and productivity of codes. For this reason, we think that combining DSLs and component-based runtimes could enable those properties, by inheritance, in DSLs.
This paper proposes a first evaluation of this approach though a use-case study that is a DSL for \emph{multi-stencil} programs, \ie a subclass of numerically solved partial differential equations (PDEs), and its compiler.
%\fix{cp2cp+hc: to dev benefits of components.}

Many solutions to automatically optimize and parallelize numerical simulations have been proposed in the literature, either as specific frameworks~\cite{CPE:CPE3494,petsc-efficient} or as DSLs~\cite{spaaTangCKLL11,citeulike12258902,Giles2011,DeVito2011LDS}. Most of the time, DSLs focus on the parallelization and the optimization of a single numerical computation, also called a stencil kernel.
However, a real case numerical simulation (such as fluid dynamics, magneto-hydrodynamics, molecular dynamics etc.) is most of the time not composed of a single stencil kernel, but of a set of stencil kernels and a set of additional local auxiliary computations~\cite{Ragan-Kelley:2013:HLC:2491956.2462176}. The DSL proposed in this paper, called \emph{Multi-Stencil Language} or MSL, is a DSL to generate a parallel component-based structure of a numerical simulation, that we call a \emph{multi-stencil program}. As discussed in the related work, this work is complementary to the optimization and parallelization of a single stencil kernel and is independent from implementation choices (as for example the form of the studied mesh).

The rest of this paper is organized as follows. The concepts of \emph{stencil kernels} and \emph{multi-stencil programs} are formalized in Section~\ref{sect:concept}. Section~\ref{sect:mscac} gives an overview of the proposed solution while Section~\ref{sect:msmsc} details the MSL language and its compiler which automatically generates the parallel computation part. Section~\ref{sect:component} introduces component models and describes the proposed component-based runtime of MSL.
%The end of Section~\ref{sect:component} also discusses on interest of this new approach.
Section~\ref{sect:eval} focuses on a real-case simulation, solving the Shallow-water equations, and analyzes preliminary performance and usability results. Sections~\ref{sect:related} states on related work, while Section~\ref{sect:conclusion} concludes the paper.
