Programming complex scientific applications on high performance heterogenous architectures, such as clusters of nodes containing multicores and accelerators, requires for the scientists to be an expert in his own domain, but also to be an expert in low level parallel and high performance programming. It can takes months to years to a scientist to learn HPC programming, as it is not his main activity. And while learning HPC programming, hardware continue to evolve, becoming more powerful but usually also more  complex and difficult to use efficiently.

Domain Specific Languages (DSLs) are a promising solution to hide intricate details of HPC programming from non experts, while producing automatically an optimized and efficient final code. Thus, it seems that a clue for simplicity and efficiency on modern architectures is to specialize a language to a given domain, from which one or more efficient parallelizations and optimizations can be extracted.
However, if the specific nature of a DSL simplifies HPC programming for scientists, its implementation cost (language, compiler, and runtime) is difficult to amortize. Actually, programming difficulties, as well as non-maintainability and non-portability seems to be deported from the scientist to DSL designers. Moreover, one can consider that end users need as much DSLs as domains, which keeps the amount of work to write and maintain up-to-date DSLs out of hands.

For this reason, solutions to ease the design and implementation of DSLs have recently appear~\cite{Fernandez:2014:DFL:2691166.2691168}. The main idea of those solutions is to propose a single DSL to write new DSLs. In this paper, we study a first contribution to another approach to bring maintainability and code-reuse in DSLs. This approach proposes a transformation from a DSL to a component-based runtime. \fix{cp2cp+hc: to dev benefits of components.}

To evaluate such an approach, a proper new DSL for \emph{multi-stencil} programs, and its compiler are presented in this paper. Many solutions to automatically optimize and parallelize partial differential equations (PDEs) solvers have been proposed in the litterature, as specific libraries~\cite{petsc-efficient,Trilinos-Overview,CPE:CPE3494}, and DSLs~\cite{spaaTangCKLL11,citeulike12258902,Giles2011,DeVito2011LDS}. Most of the time, DSLs focuss on the parallelization and the optimization of a single numerical computation, also called a stencil.
However, a real case numerical simulation is most of the time not composed of a single stencil computation, but of a set of stencil computations and a set of additionnal auxiliary local computations. The DSL proposed in this paper, called MSL, is a DSL to generate a parallel component-based structure of an overall numerical simulation, that we call a \emph{multi-stencil programs}. As it will be detailed in the related work, this work is complementary to the optimization and parallelization of a single stencil computation.

The rest of this paper is organized as follows. First concepts of \emph{stencil} and \emph{multi-stencil porgams} are formalized in Section~\ref{sect:concept}. Section~\ref{sect:mscac} gives an overview of the solution presented in this paper while Section~\ref{sect:msmsc} details the MSL language and its compiler. Section~\ref{sect:component} introduces component models and then it describes the proposed component-based runtime and the automatic generation of its parallel compuration part.
%The end of Section~\ref{sect:component} also discusses on interest of this new approach.
Section~\ref{sect:eval} focuses on a real-case simulation, solving the Shallow-water equations, and analyzes some performance results. Sections~\ref{sect:related} states on related work, and finally Section~\ref{sect:conclusion} concludes and gives some perspectives.
