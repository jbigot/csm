This paper has presented the domain specific language MSL and its compiler MSCAC designed to produce a parallel (data and hybrid) component-based runtime of an overall multi-stencil program, \ie a mesh-based numerical simulation reduced to explicit schemes. After details on the language and its compiler, MSL has been evaluated by the description of a real case numerical simulation: the shallow water equations. The component-based data parallelization of the simulation has been compared to a pure SkelGIS parallelization, and has shown improved execution times as well as a promising scalability. Those results demonstrate that component-based runtimes may be relevant back-end codes for DSLs as they do not introduce performance damage. Moreover, components bring software engineering benefits such as separation of concern, code re-use, improving maintainability.

Although MSL is a promising case study from DSLs to component-based runtimes, many works in progress aims at improving this first contribution. First, to more clearly show the improvement of DSLs maintainability using component-based back-end, an alternative \texttt{DDS} component is under study, using Global Arrays~\cite{Nieplocha:2006:AAP:1125980.1125985}. In addition to this, alternatives for the \texttt{Computations} component, computed by MSC, are under study such as a dump to an OpenMP~\cite{660313} code or the use of dynamic schedulers~\cite{Augonnet2011,Gautier:2013:XRS:2510661.2511383}. This last work on dynamic schedulers could also improve the expressivity of task parallelism in MSL, as it has been discussed in Section~\ref{sect:eval}, and thus it could bring interesting performance for the hybrid parallelization.
