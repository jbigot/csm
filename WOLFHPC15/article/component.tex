%----------------------------------------
%\subsection{Overview}
%----------------------------------------
This section describes the projection of a MSP program to components, including the components used to implement the static scheduling of $\Gamma_{tsp}$.
Before those details, the section gives an overview of component models, and especially the Low Level Components (\llc) used in this paper.


%----------------------------------------
\subsection{Component-Based Software Engineering}
%----------------------------------------
Component-based software engineering (CBSE) is a domain of software engineering~\cite{Szyperski:2002:CSB:515228} which promotes code
re-use, separations of concerns, and thus maintainability.
An application is made of a set of component instances.
A component is a black box that implements an independent functionality of the application, and which interacts with its environment only through well defined interfaces: its ports.
A port can, for example, specify services provided or required by the component.
With respect to high performance computing, some works have also shown
that component models can achieve the needed level of performance, and
scalability while also helping in application
portability~\cite{Bernholdt01052006, bigot:inria-00388508, UCHPC2015}

Many component models exist, each of them with its own specificities.
Well known component models include, for example, the CORBA Component Model (CCM)~\cite{corba:omg06}, and the Grid Component Model (GCM)~\cite{Baude} for distributed computing, while the Common Component Architecture (CCA)~\cite{Bernholdt01052006}, and Low Level Components (\llc)~\cite{l2c} are HPC-oriented.
This work makes use of \llc for the experiments.
Let us therefore introduce the concepts of this model in more details.

%----------------------------------------
\subsection{\llc}
%----------------------------------------

\llc is a minimalist \texttt{C++} based HPC-oriented component model
where a component extends the concept of class.
The services offered by the components are specified trough $provide$ ports,
those used either by $use$ ports for a single service instance,
or $use-multiple$ ports for multiple service instances.
Services are specified as \texttt{C++} interfaces.
\llc also offers $MPI$ ports that enable components to share MPI communicators.
Components can also have attribute ports to be configured.
%
As illustrated in Figures~\ref{fig:ports}, a $provide$ port is
represented by a white circle, a $use$
port with a black circle, a $use-multiple$ port by a black circle with
a white $m$ in it. MPI port are
connected with a black rectangle.
A \llc-based application is a static \emph{assembly} of components instances and connections between their ports.
Such an assembly is described in LAD, an XML dialect~\cite{l2c}.

\begin{figure}[t]
\begin{center}
\subfloat[][\label{fig:2comp}]{
\begin{tikzpicture}[shorten >=1pt, node distance=2cm, on grid, auto]
   \node[component] (C) at (0,0) {$c_0$};
   \node[provide] (p) at (-1.5,0) {};
   \node[use] (u) at (1.5,0) {};
   \node[provide,right=1.5cm of u] (p1) {};
   \node[component,right=1.5cm of p1] (C1) {$c_1$};
   \node[use,right=1.5cm of C1] (um) {$m$};
 
  \path[-]
    (p) edge node {$p$} (C)
    (C.east) edge node {$u$} (u)
    (C1)  edge node {$v$} (um)
    (p1) edge node {$q$} (C1);
\end{tikzpicture}
}
\\
\subfloat[][\label{fig:ass}]{
\begin{tikzpicture}[shorten >=1pt, node distance=2cm, on grid, auto]
   \node[component] (C) at (0,0) {$c_0$};
   \node[provide] (p) at (-1.5,0) {};
   \node[use] (u) at (1.5,0) {};
   \node[provide,right=0.15 of u] (p2) {};
   \node[component,right=1.5 of p2] (C1) {$c_1$};
   \node[use,right=1.5 of C1] (um) {$m$};
 
  \path[-]
    (p) edge node {$p$} (C)
    (C) edge node {$u$} (u)
    (C1)  edge node {$v$} (um)
    (p2) edge node {$q$} (C1);
\end{tikzpicture}
}
\\
\subfloat[][\label{fig:mpi}]{
\begin{tikzpicture}[shorten >=1pt, node distance=2cm, on grid, auto]
   \node[component] (C) at (0,0) {$c_2$};
   \node[provide] (p) at (-1.5,0) {};
   \node[mpi] (u) at (1.5,0) {};
   \node[component,right=1.5 of u] (C1) {$c_3$};
 
  \path[-]
    (p) edge node {} (C)
    (C) edge node {} (u)
    (u) edge node {} (C1);
\end{tikzpicture}
}
\caption{Example of components and their ports representation. a) Component $c_0$ has a provide port ($p$) and a use port ($u$); Component $c_1$ has also a provide port ($q$) but also a use multiple port ($v$). b) A use port is connected to a (compatible) provide port. c) Component $c_2$ and $c_3$ shares an MPI communicator.}
\label{fig:ports}
\end{center}
\end{figure}

%% In the rest of this paper, when a required service of a \emph{use} (or \emph{use-multiple}) port is filled and linked to a \emph{provide} port in the comonent assembly, only the use port stay visible, as illustrated in Figure~\ref{fig:assembly}. As the use port is on the right of its component, and the provide port on the left, in a component assembly the component on the left uses the provide port of the component on the right.

%----------------------------------------
\subsection{MSCAC Runtime Overview}
%----------------------------------------

\begin{figure}[t]
\begin{center}
\begin{tikzpicture}[shorten >=1pt, node distance=2cm, on grid, auto]
   \node[component] (D) at (0,0) {$Driver$};
   \node[provide] (Dp) at (-1,0) {};
   \node (Ds) at (-1.5,0) {start};
   \node[use,right=1.5cm of D] (Du1) {};
   \node[use,below=1.75cm of D] (Du2) {};
   \node[use,right=1.7cm of Du2] (Du3) {$m$};

   \node[provide,below=0.15 of Du2] (Tp) {};
   \node[component,below=1.6cm of Tp] (T) {$Time$};
   \node[use,right=1cm of T] (Tu) {};
   \node[left=0.8cm of T] (tt) {$T$};

   \node[provide,right=0.15 of Tu] (Cp) {};
   \node[component,right=2cm of Cp] (C) {$Computations$};
   \node[use,above=0.8cm of C] (Cu) {};
   \node[right=0.2cm of Cu] (star) {$*$};
   \node[right=1.5cm of C] (gamma) {$\Gamma$};

   \node[provide,right=0.2 of Du3] (Datap1) {};
   \node[provide,above=0.15 of Cu] (Datap2) {};
   \node[component,above=0.8cm of Datap2] (Data) {$Data$};
   \node[use,above=0.8cm of Data] (Datau) {};
   \node[right=1cm of Data] (delta) {$\Delta,\mathcal{D}$};
   \node[mpi,above right=1cm of Data] (mpidata) {};

   \node[provide,right=0.15 of Du1] (DDSp1) {};
   \node[provide,above=0.15 of Datau] (DDSp2) {};
   \node[component,above=0.8cm of DDSp2] (DDS) {$DDS$};
   \node[right=1.2cm of DDS] (m) {$\mathcal{M},\mathcal{E}$};
 
  \path[-]
    (Dp) edge node {} (D)
    (D) edge node {} (Du1)
        edge node {} (Du2)
        edge node {} (Du3)
    (DDSp1) edge node {} (DDS)
    (Tp) edge node {} (T)
    (T)  edge node {} (Tu)
    (Cp) edge node {} (C)
    (C) edge node {} (Cu)
    (Datap2) edge node {} (Data)
    (Datap1) edge node {} (Data)
    (Data) edge node {} (Datau)
          edge node {} (mpidata)
    (DDSp2) edge node {} (DDS);
\end{tikzpicture}
\vspace*{.5em}
\caption{MSCAC Component Assembly Back-End Overview.}
\label{fig:mscac:assembly}
\end{center}
\end{figure}

As described in Section~\ref{sect:mscac}, the compiler generates four independent parts in a component assembly in addition to a static part.
This is represented on Figure~\ref{fig:mscac:assembly} where each rounded box represents one or more component instances.
\texttt{Driver} is static and drives the whole application.
The four other parts are generated based on the content of sections of the MSL program: \texttt{Computations} from $\Gamma$ (with code provided by the user), \emph{DDS} from $\mathcal{M}$ and $\mathcal{E}$, \emph{Data} from $\Delta$ and $\mathcal{D}$, and \emph{Time} from $T$.

\texttt{DDS} manages the structure of the mesh, a single instance is generated as MSL only supports a single mesh for now.
The back-end presented in this paper uses the distributed data structure proposed in SkelGIS library~\cite{CPE:CPE3494}.
\texttt{DDS} is used by \texttt{Data} containing one component instance for each data element in the MSL program.
\texttt{Time} is parametrized by the the \texttt{iteration} value from MSL and calls the \texttt{Computations} part as many times as required.

\texttt{Computations} contains the static scheduling computed by MSC, \ie $\Gamma_{tsp}$, or the data parallel transformation $\Gamma_{data}$.
Both transformations $\Gamma_{tsp}$ and $\Gamma_{data}$ are encoded using three specific components which manage the $P$, $S$, and $sync$ operations as explained hereafter.
At the leaves of the tree, the computation components instances ($K$), provided by the user, are connected to \texttt{Data}.
As presented on Figure~\ref{fig:k}, these components provide a single port used to launch the computation and expose one use port for each data element manipulated in the computation.
We denote the fact that they may have several use port by adding a star on the use port.
Each of these use ports is connected to the matching component instance in \texttt{Data}.
We denote this kind of component K.

As a result of this approach, each part of the generated code is rather independent.
Changing the main loop to use a convergence criteria rather than a fixed number of iterations would only require changing the \texttt{Time} component.
Similarly, changing the approach used for scheduling would only impact the code generated in \texttt{Computations}.
The technology used for data parallelism can also be changed by replacing the \texttt{DDS} and \texttt{Data} components; as a matter of fact a work in progress proposes an alternative based on Global Arrays~\cite{Nieplocha:2006:AAP:1125980.1125985} instead of SkelGIS.
In this case however, the kernel components of the \texttt{Computations} part, provided by the user, might have to be written with interfaces of the alternative distributed data structure.

\begin{figure}
\begin{center}
\begin{tikzpicture}[shorten >=1pt, node distance=2cm, on grid, auto]
   \node[component] (k) at (0,0) {$K$};
   \node[provide] (p) at (-1.5,0) {};
   \node[use] (u) at (1.5,0) {};
   \node[right=0.2 of u] (star) {*};
 
  \path[-]
    (p) edge node {} (k)
    (k) edge node {} (u);
\end{tikzpicture}
\end{center}
\caption{Kernel component.}
\label{fig:k}
\end{figure}
  
%----------------------------------------
\subsection{Control Components and Dump}
%----------------------------------------
The series-parallel tree decomposition $\Gamma_{tsp}$ represents a static scheduling of the simulation.
It is dumped to components by generating an assembly that exactly matches the tree, as well as for $\Gamma_{data}$.
We introduce what we call \emph{control components} to represent all nodes of $\Gamma_{tsp}$ or $\Gamma_{data}$.
These components can be used for any simulation, which increases code reuse between simulations.
A control component exposes a single \emph{provide} port containing a single method, typically a control-only method without any parameter.
We introduce three control components represented in Figure~\ref{fig:ctrlcomponents}.

\begin{figure}[t]
\begin{tikzpicture}[shorten >=1pt, node distance=1cm, on grid, auto]
   \node[component] (seq) at (0,0) {$SEQ$}; \node[provide] (p) at
   (-1,0) {}; \node[use] (u) at (1,0) {$m$};
 
  \path[-]
    (p) edge node {} (seq)
    (seq) edge node {} (u);
\end{tikzpicture}
\hspace{\fill}
\begin{tikzpicture}[shorten >=1pt, node distance=1cm, on grid, auto]
   \node[component] (seq) at (0,0) {$PAR$};
   \node[provide] (p) at (-1,0) {};
   \node[use] (u) at (1,0) {$m$};
 
  \path[-]
    (p) edge node {} (seq)
    (seq) edge node {} (u);
\end{tikzpicture}
\hspace{\fill}
\begin{tikzpicture}[shorten >=1pt, node distance=1cm, on grid, auto]
   \node[component] (sync) at (0,0) {$SYNC$};
   \node[provide] (p) at (-1,0) {};
   \node[use] (u) at (1,0) {};
 
  \path[-]
    (p) edge node {} (sync)
    (sync) edge node {} (u);
\end{tikzpicture}
\\
a) Comp. SEQ
\hspace{\fill}
b) Comp. PAR
\hspace{\fill}
c) Comp. SYNC
\\
\caption{The three control components used by the back-end.}
\label{fig:ctrlcomponents}
\end{figure}

\begin{description}
\item[Sequence component (SEQ)] It is the direct representation of a sequence node of $\Gamma_{tsp}$. This component sequentially calls an ordered list of other components. It exposes an ordered \emph{use-multiple} port to be connected to the components to call in sequence.

\item[Parallel component (PAR)] It is the direct representation of a parallel node of $\Gamma_{tsp}$. This component simultaneously calls a set of other components. It exposes a \emph{use-multiple} port to be connected to the components to call in parallel.
  
\item[Synchronization component (SYNC)]. It is the direct representation to an update leaf of $\Gamma_{tsp}$. This component calls the synchronization of a given data. It exposes a \emph{use} port to be connected to the data to update.

\end{description}


%----------------------------------------
%\subsection{Dump To Component Assembly}
%----------------------------------------
Then from $\Gamma_{tsp}$ or $\Gamma_{data}$, and using the control components described above,
a direct dump can be done to a component assembly for a data or hybrid (data
and task) parallelization of a simulation.
%However, it is also
%possible to dump to a data parallelization only. In such a case the
%computation of $\Gamma_{sync}$ is enough to generate the component
%assembly. Actually a single SEQ component is thus needed and this
%component is linked to all computations and synchronizations of
%$\Gamma_{sync}$ directly.

Figure~\ref{fig:tsp:assembly} displays the assembly part corresponding to
$\Gamma_{tsp}$ of Figure~\ref{fig:tsp}. In this figure, the ports
linked to data (use and use-multiple ports of SYNC and K) are
represented but are not connected. Moreover, each computation
component is an instance of a specific K component using
the identification name of the computation in the MSL file.

\begin{figure*}[t]
\begin{center}
\begin{tikzpicture}[shorten >=1pt, node distance=2cm, on grid, auto]
   %seq0
   \node[component] (seq0) at (0,0) {$SEQ$};
   \node[provide] (seq0p) [left = 0.8cm of seq0] {};
   \node[use] (seq0u) [right = 1cm of seq0] {$m$};
   %c0
   \node[component] (c0) [right = 2cm of seq0u] {$K(c_0)$};
   \node[use] (c0u) [right = 0.8cm of c0] {};
   \node[right=0.2 of c0u] (star) {*};
   %sync0
   \node[component] (sync0) [below = 1cm of c0] {$SYNC$};
   \node[use] (sync0u) [right = 1cm of sync0] {$m$};
   %c1
   \node[component] (c1) [below = 0.8cm of sync0] {$K(c_1)$};
   \node[use] (c1u) [right = 0.8cm of c1] {};
   \node[right=0.2 of c1u] (star) {*};
   %par0
   \node[component] (par0) [below = 0.8cm of c1] {$PAR$};
   \node[use] (par0u) [right = 1cm of par0] {$m$};
  %seq1
   \node[component] (seq1) [right = 1cm of sync0u] {$SEQ$};
   \node[use] (seq1u) [right = 1cm of seq1] {$m$};
  %c2
   \node[component] (c2) [right = 4cm of c0] {$K(c_2)$};
   \node[use] (c2u) [right = 0.8cm of c2] {};
   \node[right=0.2 of c2u] (star) {*};
  %sync1
   \node[component] (sync1) [below = 0.8cm of c2] {$SYNC$};
   \node[use] (sync1u) [right = 1cm of sync1] {$m$};
  %c4
   \node[component] (c4) [below = 0.8cm of sync1] {$K(c_4)$};
   \node[use] (c4u) [right = 0.8cm of c4] {};
   \node[right=0.2 of c4u] (star) {*};
  %c6
   \node[component] (c6) [below = 0.8cm of c4] {$K(c_6)$};
   \node[use] (c6u) [right = 0.8cm of c6] {};
   \node[right=0.2 of c6u] (star) {*};
  %seq2
   \node[component] (seq2) [below = 3.2cm of seq1] {$SEQ$};
   \node[use] (seq2u) [right = 1cm of seq2] {$m$};
  %c3
   \node[component] (c3) [right = 1cm of seq2u] {$K(c_3)$};
   \node[use] (c3u) [right = 0.8cm of c3] {};
   \node[right=0.2 of c3u] (star) {*};
  %c5
   \node[component] (c5) [below = 0.8cm of c3] {$K(c_5)$};
   \node[use] (c5u) [right = 0.8cm of c5] {};
   \node[right=0.2 of c5u] (star) {*};
  %c7
   \node[component] (c7) [below = 0.8cm of par0] {$K(c_7)$};
   \node[use] (c7u) [right = 0.8cm of c7] {};
   \node[right=0.2 of c7u] (star) {*};
  %sync2
   \node[component] (sync2) [below = 0.8cm of c7] {$SYNC$};
   \node[use] (sync2u) [right = 1cm of sync2] {$m$};
  %c8
   \node[component] (c8) [below = 0.8cm of sync2] {$K(c_8)$};
   \node[use] (c8u) [right = 0.8cm of c8] {};
   \node[right=0.2 of c8u] (star) {*};

   \path[-]
   %seq0
    (seq0) edge node {} (seq0u)
    (seq0p) edge node {} (seq0)
   %c0
    (seq0u) edge node {} (c0)
    (c0) edge node {} (c0u)
    %sync0
    (seq0u) edge node {} (sync0.west)
    (sync0) edge node {} (sync0u)
    %c1
    (seq0u) edge node {} (c1)
    (c1) edge node {} (c1u)
   %par0
    (seq0u) edge node {} (par0.west)
    (par0) edge node {} (par0u)
  %seq1
    (par0u) edge node {} (seq1.west)
    (seq1) edge node {} (seq1u)
  %c2
    (seq1u) edge node {} (c2.west)
    (c2) edge node {} (c2u)
  %sync0
    (seq1u) edge node {} (sync1.west)
    (sync1) edge node {} (sync1u)
  %c4
    (seq1u) edge node {} (c4.west)
    (c4) edge node {} (c4u)
  %c6
    (seq1u) edge node {} (c6.west)
    (c6) edge node {} (c6u)
  %seq2
    (par0u) edge node {} (seq2.west)
    (seq2) edge node {} (seq2u)
  %c3
    (seq2u) edge node {} (c3)
    (c3) edge node {} (c3u)
  %c6
    (seq2u) edge node {} (c5.west)
    (c5) edge node {} (c5u)
  %c6
    (seq0u) edge node {} (c7.west)
    (c7) edge node {} (c7u)
  %sync2
    (seq0u) edge node {} (sync2.west)
    (sync2) edge node {} (sync2u)
  %c8
    (seq0u) edge [bend right] node {} (c8.west)
    (c8) edge node {} (c8u)
   ;
\end{tikzpicture}
\vspace*{.5em}
\caption{Component assembly representing the computation part generated from $\Gamma_{tsp}$ of Figure~\ref{fig:tsp}.}
\label{fig:tsp:assembly}
\end{center}
\end{figure*}

%----------------------------------------
