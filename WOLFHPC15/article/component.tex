%----------------------------------------
%\subsection{Overview}
%----------------------------------------
This section describes the projection of a MSL program to a component-based runtime, including the components used to implement the static scheduling of $\Gamma_{tsp}$.
The section first gives an overview of component models, and especially the Low Level Components (\llc) used in this work.


%----------------------------------------
\subsection{Component-Based Software Engineering}
%----------------------------------------
Component-based software engineering (CBSE) is a domain of software engineering~\cite{Szyperski:2002:CSB:515228} which promotes code
re-use, separations of concerns, and thus maintainability.
An application is made of a set of component instances.
A component is a black box that implements an independent functionality of the application, and which interacts with its environment only through well defined interfaces: its ports.
A port can, for example, specify services provided or required by the component.
With respect to high performance computing, some works have also shown
that component models can achieve the needed level of performance, and
scalability while also helping in application
portability~\cite{Bernholdt01052006, bigot:inria-00388508, UCHPC2015}

Many component models exist, each of them with its own specificities.
Well known component models include, for example, the CORBA Component Model (CCM)~\cite{corba:omg06}, and the Grid Component Model (GCM)~\cite{Baude} for distributed computing, while the Common Component Architecture (CCA)~\cite{Bernholdt01052006}, and Low Level Components (\llc)~\cite{l2c} are HPC-oriented.
This work makes use of \llc for the experiments.
%Let therefore introduce this model in more details.

%----------------------------------------
\subsection{\llc}
%----------------------------------------

\llc~\cite{l2c} is a minimalist \texttt{C++} based HPC-oriented component model
where a component extends the concept of class.
The services offered by the components are specified trough $provide$ ports,
those used either by $use$ ports for a single service instance,
or $use-multiple$ ports for multiple service instances.
Services are specified as \texttt{C++} interfaces.
\llc also offers $MPI$ ports that enable components to share MPI communicators.
Components can also have attribute ports to be configured.
%
As illustrated in Figures~\ref{fig:ports}, a $provide$ port is
represented by a white circle, a $use$
port with a black circle, a $use-multiple$ port by a black circle with
a white $m$ in it. MPI port are
connected with a black rectangle.
A \llc-based application is a static \emph{assembly} of components instances and connections between their ports.
Such an assembly is described in LAD, an XML dialect.

\begin{figure}[t]
\begin{center}
\subfloat[][\label{fig:2comp}]{
\begin{tikzpicture}[shorten >=1pt, node distance=2cm, on grid, auto]
   \node[component] (C) at (0,0) {$c_0$};
   \node[provide] (p) at (-1.5,0) {};
   \node[use] (u) at (1.5,0) {};
   \node[provide,right=1.5cm of u] (p1) {};
   \node[component,right=1.5cm of p1] (C1) {$c_1$};
   \node[use,right=1.5cm of C1] (um) {$m$};
 
  \path[-]
    (p) edge node {$p$} (C)
    (C.east) edge node {$u$} (u)
    (C1)  edge node {$v$} (um)
    (p1) edge node {$q$} (C1);
\end{tikzpicture}
}
\\
\subfloat[][\label{fig:ass}]{
\begin{tikzpicture}[shorten >=1pt, node distance=2cm, on grid, auto]
   \node[component] (C) at (0,0) {$c_0$};
   \node[provide] (p) at (-1.5,0) {};
   \node[use] (u) at (1.5,0) {};
   \node[provide,right=0.15 of u] (p2) {};
   \node[component,right=1.5 of p2] (C1) {$c_1$};
   \node[use,right=1.5 of C1] (um) {$m$};
 
  \path[-]
    (p) edge node {$p$} (C)
    (C) edge node {$u$} (u)
    (C1)  edge node {$v$} (um)
    (p2) edge node {$q$} (C1);
\end{tikzpicture}
}
\\
\subfloat[][\label{fig:mpi}]{
\begin{tikzpicture}[shorten >=1pt, node distance=2cm, on grid, auto]
   \node[component] (C) at (0,0) {$c_2$};
   \node[provide] (p) at (-1.5,0) {};
   \node[mpi] (u) at (1.5,0) {};
   \node[component,right=1.5 of u] (C1) {$c_3$};
 
  \path[-]
    (p) edge node {} (C)
    (C) edge node {} (u)
    (u) edge node {} (C1);
\end{tikzpicture}
}
\caption{Example of components and their ports representation. a) Component $c_0$ has a provide port ($p$) and a use port ($u$); Component $c_1$ has also a provide port ($q$) but also a use multiple port ($v$). b) A use port is connected to a (compatible) provide port. c) Component $c_2$ and $c_3$ shares an MPI communicator.}
\label{fig:ports}
\end{center}
\end{figure}

%% In the rest of this paper, when a required service of a \emph{use} (or \emph{use-multiple}) port is filled and linked to a \emph{provide} port in the comonent assembly, only the use port stay visible, as illustrated in Figure~\ref{fig:assembly}. As the use port is on the right of its component, and the provide port on the left, in a component assembly the component on the left uses the provide port of the component on the right.

%----------------------------------------
\subsection{MSP Component Runtime Overview}
%----------------------------------------

\begin{figure}[t]
\begin{center}
\begin{tikzpicture}[shorten >=1pt, node distance=2cm, on grid, auto]
   \node[component] (D) at (0,0) {$Driver$};
   \node[provide] (Dp) at (-1,0) {};
   \node (Ds) at (-1.5,0) {start};
   \node[use,right=1.5cm of D] (Du1) {};
   \node[use,below=1.75cm of D] (Du2) {};
   \node[use,below=0.8cm of Du1] (Du3) {$m$};

   \node[provide,below=0.15 of Du2] (Tp) {};
   \node[component,below=1.6cm of Tp] (T) {$Time$};
   \node[use,right=1cm of T] (Tu) {};
   \node[left=0.8cm of T] (tt) {$T$};

   \node[provide,right=0.15 of Tu] (Cp) {};
   \node[component,right=2cm of Cp] (C) {$Computations$};
   \node[use,above=0.8cm of C] (Cu) {};
   \node[right=0.2cm of Cu] (star) {$*$};
   \node[right=1.5cm of C] (gamma) {$\Gamma$};

   \node[provide,below right=0.2 of Du3] (Datap1) {};
   \node[provide,above=0.15 of Cu] (Datap2) {};
   \node[component,double,above=0.8cm of Datap2] (Data) {$Data$};
   \node[use,above=0.8cm of Data] (Datau) {};
   \node[left=1cm of Data] (delta) {$\Delta,\mathcal{D}$};
   \node[mpi,right=1cm of Data] (mpidata) {};
   \node[above=0.2cm of mpidata] (star2) {$*$};

   \node[provide,right=0.15 of Du1] (DDSp1) {};
   \node[provide,above=0.15 of Datau] (DDSp2) {};
   \node[component,above=0.8cm of DDSp2] (DDS) {$DDS$};
   \node[right=1.2cm of DDS] (m) {$\mathcal{M},\mathcal{E}$};
 
  \path[-]
    (Dp) edge node {} (D)
    (D) edge node {} (Du1)
        edge node {} (Du2)
        edge node {} (Du3)
    (DDSp1) edge node {} (DDS)
    (Tp) edge node {} (T)
    (T)  edge node {} (Tu)
    (Cp) edge node {} (C)
    (C) edge node {} (Cu)
    (Datap2) edge node {} (Data)
    (Datap1) edge node {} (Data)
    (Data) edge node {} (Datau)
          edge node {} (mpidata)
    (DDSp2) edge node {} (DDS);
\end{tikzpicture}
\vspace*{.5em}
\caption{MSCAC Component-based Runtime Overview.}
\label{fig:mscac:assembly}
\end{center}
\end{figure}

As described in Section~\ref{sect:mscac}, the compiler generates four independent parts in a component assembly in addition to a static part as represented in Figure~\ref{fig:mscac:assembly}: each rounded box represents one component instance, drawn with simple lines, or more, drawn with double lines. For example, the component \texttt{Data} is instantiated more than once.
\texttt{Driver} is static and drives the whole application.
The four other parts are generated based on the content of the different sections of the MSL program: \texttt{Computations}, from the parallelization of $\Gamma$ (done by MSC as explained in Section~\ref{sect:msc}), \emph{DDS} from $\mathcal{M}$ and $\mathcal{E}$, \emph{Data} from $\Delta$ and $\mathcal{D}$, and \emph{Time} from $T$.

%, and using the kernel codes provided by the user

\texttt{DDS} manages the structure and the partitioning of the mesh, a single instance is generated as MSL only supports a single mesh for now.
The back-end presented in this paper uses the distributed data structure proposed in SkelGIS library~\cite{CPE:CPE3494}.
\texttt{DDS} is used by \texttt{Data}, for which one instance represents one data element in the MSL program.
\texttt{Time} is parametrized by the \texttt{iteration} value from MSL and uses the \texttt{Computations} component as many times as required.

\texttt{Computations} contains the static scheduling $\Gamma_{tsp}$, or the data parallel transformation $\Gamma_{data}$, both computed by MSC.
Both transformations, $\Gamma_{tsp}$ and $\Gamma_{data}$, are encoded using three specific components, detailed in Section~\ref{sect:control}, which manage the $P$, $S$, and $sync$ operations.
At the leaves of $\Gamma_{tsp}$ and $\Gamma_{data}$, the computation components instances (denoted $K$), provided by the user, are connected to \texttt{Data}.
As shown in Figure~\ref{fig:k}, a $K$ component provides a port to launch the computation and exposes a use port for each data element manipulated in the computation; a star is added on the use port to \texttt{Data} to denote it.
\begin{figure}
\begin{center}
\begin{tikzpicture}[shorten >=1pt, node distance=2cm, on grid, auto]
   \node[component] (k) at (0,0) {$K$};
   \node[provide] (p) at (-1.5,0) {};
   \node[use] (u) at (1.5,0) {};
   \node[right=0.2 of u] (star) {*};
 
  \path[-]
    (p) edge node {} (k)
    (k) edge node {} (u);
\end{tikzpicture}
\end{center}
\caption{A kernel component $K$.}
\label{fig:k}
\end{figure}


A complete assembly is obtained by replicating the component assembly of Figure~\ref{fig:mscac:assembly} on available processors (or cores). Connections is done with MPI ports of \texttt{Data} components.

As a result of this approach, each part of the generated code is rather independent.
Changing the main time loop to use a convergence criteria rather than a fixed number of iterations would only require changing the \texttt{Time} component.
Similarly, changing the approach used for scheduling would only impact the code generated in \texttt{Computations}.
The technology used for data parallelism can also be changed by replacing the \texttt{DDS} and \texttt{Data} components, and by using the adequate interfaces in $K$. As a matter of fact, we have started evaluating an alternative based on Global Arrays~\cite{Nieplocha:2006:AAP:1125980.1125985} instead of SkelGIS.
  
%----------------------------------------
\subsection{Control Components and Dump}
\label{sect:control}
%----------------------------------------
The series-parallel tree decomposition $\Gamma_{tsp}$ represents a static scheduling of the simulation. $\Gamma_{data}$, on the other hand, is an optimized merged ordered list of computations. Both are dumped to components by generating an assembly that exactly matches their structure.
We introduce what we call \emph{control components} to represent all nodes of $\Gamma_{tsp}$ or $\Gamma_{data}$.
These components can be used for any simulation, which increases code reuse between simulations.
A control component exposes a single \emph{provide} port containing a \emph{control-only} method, without any parameter.
Three control components have been needed: SEQ, PAR, and SYNC. They are represented in Figure~\ref{fig:ctrlcomponents}.


\begin{description}
\item[Sequence component (SEQ)] It is the direct representation of a sequence node of $\Gamma_{tsp}$ or $\Gamma_{data}$. This component sequentially calls an ordered list of other components. It exposes an ordered \emph{use-multiple} port to be connected to the components to call in sequence.

\item[Parallel component (PAR)] It is the direct representation of a parallel node of $\Gamma_{tsp}$ (not $\Gamma_{data}$). This component simultaneously calls a set of other components. It exposes a \emph{use-multiple} port to be connected to the components to call in parallel.
  
\item[Synchronization component (SYNC)] It is the direct representation to an update leaf of $\Gamma_{tsp}$ or $\Gamma_{data}$. This component calls the synchronization of a given data. It exposes a \emph{use} port to be connected to the data to be updated.
\end{description}

Then, from $\Gamma_{tsp}$ or $\Gamma_{data}$, and using the control components described above,
a direct dump can be done to a component assembly for a data or hybrid (data
and task) parallelization of a simulation.
Figure~\ref{fig:tsp:assembly} displays the assembly part corresponding to
$\Gamma_{tsp}$ of Figure~\ref{fig:tsp}. In this figure, the ports
linked to data (use and use-multiple ports of SYNC and K) are
represented but are not connected for readability. Moreover, each computation
component is an instance of a specific K component using
the identification name of the computation of an MSL instance.
Thus, this figure represents what the component \texttt{Computations} of Figure~\ref{fig:mscac:assembly} contains if an hybrid parallelization is performed. On the other hand, if a data parallelization is performed, a single $SEQ$ component is used by \texttt{Time}, which invokes in the appropriate order the computation kernels and synchronizations. Among those kernels, a single \emph{merged} kernel would represent $c_3$ and $c_5$.

\begin{figure}[t]
\begin{tikzpicture}[shorten >=1pt, node distance=1cm, on grid, auto]
   \node[component] (seq) at (0,0) {$SEQ$}; \node[provide] (p) at
   (-1,0) {}; \node[use] (u) at (1,0) {$m$};
 
  \path[-]
    (p) edge node {} (seq)
    (seq) edge node {} (u);
\end{tikzpicture}
\hspace{\fill}
\begin{tikzpicture}[shorten >=1pt, node distance=1cm, on grid, auto]
   \node[component] (seq) at (0,0) {$PAR$};
   \node[provide] (p) at (-1,0) {};
   \node[use] (u) at (1,0) {$m$};
 
  \path[-]
    (p) edge node {} (seq)
    (seq) edge node {} (u);
\end{tikzpicture}
\hspace{\fill}
\begin{tikzpicture}[shorten >=1pt, node distance=1cm, on grid, auto]
   \node[component] (sync) at (0,0) {$SYNC$};
   \node[provide] (p) at (-1,0) {};
   \node[use] (u) at (1,0) {};
 
  \path[-]
    (p) edge node {} (sync)
    (sync) edge node {} (u);
\end{tikzpicture}
\\
a) Comp. SEQ
\hspace{\fill}
b) Comp. PAR
\hspace{\fill}
c) Comp. SYNC
\\
\caption{The three control components.}
\label{fig:ctrlcomponents}
\end{figure}

\begin{figure*}[t]
\begin{center}
\begin{tikzpicture}[shorten >=1pt, node distance=2cm, on grid, auto]
   %seq0
   \node[component] (seq0) at (0,0) {$SEQ$};
   \node[provide] (seq0p) [left = 0.8cm of seq0] {};
   \node[use] (seq0u) [right = 1cm of seq0] {$m$};
   %c0
   \node[component] (c0) [right = 2cm of seq0u] {$K(c_0)$};
   \node[use] (c0u) [right = 0.8cm of c0] {};
   \node[right=0.2 of c0u] (star) {*};
   %sync0
   \node[component] (sync0) [below = 1cm of c0] {$SYNC$};
   \node[use] (sync0u) [right = 1cm of sync0] {$m$};
   %c1
   \node[component] (c1) [below = 0.8cm of sync0] {$K(c_1)$};
   \node[use] (c1u) [right = 0.8cm of c1] {};
   \node[right=0.2 of c1u] (star) {*};
   %par0
   \node[component] (par0) [below = 0.8cm of c1] {$PAR$};
   \node[use] (par0u) [right = 1cm of par0] {$m$};
  %seq1
   \node[component] (seq1) [right = 1cm of sync0u] {$SEQ$};
   \node[use] (seq1u) [right = 1cm of seq1] {$m$};
  %c2
   \node[component] (c2) [right = 4cm of c0] {$K(c_2)$};
   \node[use] (c2u) [right = 0.8cm of c2] {};
   \node[right=0.2 of c2u] (star) {*};
  %sync1
   \node[component] (sync1) [below = 0.8cm of c2] {$SYNC$};
   \node[use] (sync1u) [right = 1cm of sync1] {$m$};
  %c4
   \node[component] (c4) [below = 0.8cm of sync1] {$K(c_4)$};
   \node[use] (c4u) [right = 0.8cm of c4] {};
   \node[right=0.2 of c4u] (star) {*};
  %c6
   \node[component] (c6) [below = 0.8cm of c4] {$K(c_6)$};
   \node[use] (c6u) [right = 0.8cm of c6] {};
   \node[right=0.2 of c6u] (star) {*};
  %seq2
   \node[component] (seq2) [below = 3.2cm of seq1] {$SEQ$};
   \node[use] (seq2u) [right = 1cm of seq2] {$m$};
  %c3
   \node[component] (c3) [right = 1cm of seq2u] {$K(c_3)$};
   \node[use] (c3u) [right = 0.8cm of c3] {};
   \node[right=0.2 of c3u] (star) {*};
  %c5
   \node[component] (c5) [below = 0.8cm of c3] {$K(c_5)$};
   \node[use] (c5u) [right = 0.8cm of c5] {};
   \node[right=0.2 of c5u] (star) {*};
  %c7
   \node[component] (c7) [below = 0.8cm of par0] {$K(c_7)$};
   \node[use] (c7u) [right = 0.8cm of c7] {};
   \node[right=0.2 of c7u] (star) {*};
  %sync2
   \node[component] (sync2) [below = 0.8cm of c7] {$SYNC$};
   \node[use] (sync2u) [right = 1cm of sync2] {$m$};
  %c8
   \node[component] (c8) [below = 0.8cm of sync2] {$K(c_8)$};
   \node[use] (c8u) [right = 0.8cm of c8] {};
   \node[right=0.2 of c8u] (star) {*};

   \path[-]
   %seq0
    (seq0) edge node {} (seq0u)
    (seq0p) edge node {} (seq0)
   %c0
    (seq0u) edge node {} (c0)
    (c0) edge node {} (c0u)
    %sync0
    (seq0u) edge node {} (sync0.west)
    (sync0) edge node {} (sync0u)
    %c1
    (seq0u) edge node {} (c1)
    (c1) edge node {} (c1u)
   %par0
    (seq0u) edge node {} (par0.west)
    (par0) edge node {} (par0u)
  %seq1
    (par0u) edge node {} (seq1.west)
    (seq1) edge node {} (seq1u)
  %c2
    (seq1u) edge node {} (c2.west)
    (c2) edge node {} (c2u)
  %sync0
    (seq1u) edge node {} (sync1.west)
    (sync1) edge node {} (sync1u)
  %c4
    (seq1u) edge node {} (c4.west)
    (c4) edge node {} (c4u)
  %c6
    (seq1u) edge node {} (c6.west)
    (c6) edge node {} (c6u)
  %seq2
    (par0u) edge node {} (seq2.west)
    (seq2) edge node {} (seq2u)
  %c3
    (seq2u) edge node {} (c3)
    (c3) edge node {} (c3u)
  %c6
    (seq2u) edge node {} (c5.west)
    (c5) edge node {} (c5u)
  %c6
    (seq0u) edge node {} (c7.west)
    (c7) edge node {} (c7u)
  %sync2
    (seq0u) edge node {} (sync2.west)
    (sync2) edge node {} (sync2u)
  %c8
    (seq0u) edge [bend right] node {} (c8.west)
    (c8) edge node {} (c8u)
   ;
\end{tikzpicture}
\vspace*{.5em}
\caption{Component assembly representing the computation part generated from $\Gamma_{tsp}$ of Figure~\ref{fig:tsp}.}
\label{fig:tsp:assembly}
\end{center}
\end{figure*}

%----------------------------------------
