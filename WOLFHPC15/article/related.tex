%----------------------------------------
\subsection{Stencil solutions}
%----------------------------------------

Many solutions exists to ease parallel programming of stencil codes, as for example PATUS~\cite{citeulike12258902}, Pochoir~\cite{spaaTangCKLL11}, OP2~\cite{Giles2011}. Those solutions are powerfull, let the user implement their own stencil codes, and produce high performance codes. Those solutions can be considered as stencil compilers. As a result it produces optimized (cache tiling etc.) or parallel (CUDA, OpenCL etc.) codes for a single stencil computations. 

A real case numerical simulation is most of the time composed of a set of numerical schemes (stencils computations), with one or more stencil shape (neighboorhood shape), and onto one or more data. Moreover, a numerical simulation also performs additionnal auxiliary computations which do not involve neighborhood, called local computations. Solutions like Pochoir or PATUS consider the parallelization of a single stencil code, considering that stencil codes represent the main computation time of numerical simulations. However, by not taking into account the parallelization of the overall numerical simulation, those compilers also reduce the codes to shared memory systems. Actually using a distributed memory system, the MPI (Message Passing Interface)~\cite{Graham2009MSE} library still have to be used by the user, as well as the distribution of the mesh onto the different distant processors.

%----------------------------------------
\subsection{Component models and control}
%----------------------------------------

%----------------------------------------
\subsection{Languages and component models}
%----------------------------------------