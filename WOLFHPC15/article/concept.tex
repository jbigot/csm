%----------------------------------------
\subsection{Multi-stencil programs}
\label{sect:multistencil}
%----------------------------------------
\begin{itemize}
\item textual introduction on the problem, what is a stencil + example. What is a real simulation (more than one stencil and additionnal computations). 
\item Existing solutions with prgramming freedom (instead of PETSc and Trilinos) = stencil compilers which give optimized code for one stencil code. Thus, for distributed memory and HPC, still have to use MPI between optimized codes to get an overall parallelization of the simulation.
\item Short definition of a multi-stencil program and its computations (cite research report for more details)
\end{itemize}

%----------------------------------------
\subsection{Parallelization techniques}
\label{sect:parallel}
%----------------------------------------
\begin{itemize}
\item data parallelism
\item task parallelism
\item hybrid parallelism
\end{itemize}

%----------------------------------------
\subsection{Component models}
%----------------------------------------
Component model is an interesting domain of software engineering~\cite{Szyperski:2002:CSB:515228} which improves code re-use, scalability and maintainability of applications~\cite{Szyperski:2002:CSB:515228,bigot:inria-00388508}. Moreover, component models bring a separation of concerns in the application by a clear division of functionnalities in different components. In addition to this, recent work on component models have shown a simultaneous answer to performance, maintainability and portability of applications~\cite{l2c}, which makes this domain an interesting solution to bring maintainability and portability in HPC programming.

Component-based software engineering~\cite{Szyperski:2002:CSB:515228} extends the concept of class by specifying in its interfaces not only the services it offers, or public methods, but also all its possible interactions with outer world, including the services it requires. As a result, each component is an independant entity composed of a set of services its provides, and a set of services it requires (and uses). 

A \emph{port} is an entity embbedded in the component which makes possible an \emph{assembly} of components. An assembly of components (of instanciated components) is a way to actually connect components together and to produce a complete application, as a set of components and their interactions. A provided service inside a component is associated to a \emph{provide} port, while a required service is associated to a \emph{use} port. It is also possible to group more than one required service into a single port, as a list, called a \emph{use-multiple} port. As illustrated in Figures~\ref{fig:ports}, a provide port will be represented by a white circle, a use port by a black circle and a use-multiple port by a black circle with a white $m$ in it.

\begin{figure}[h!]
\begin{center}
\begin{tikzpicture}[shorten >=1pt, node distance=2cm, on grid, auto]
   \node[component] (C) at (0,0) {$Comp_0$};
   \node[provide] (p) at (-1.5,0) {};
   \node[use] (u) at (1.5,0) {};
   \node[use] (um) at (0,1) {$m$};
   \node[provide,right=of u] (p1) {};
   \node[component,right=of p1] (C1) {$Comp_1$};
 
  \path[-]
    (p) edge node {} (C)
    (C) edge node {} (u)
    	edge node {} (um)
    (p1) edge node {} (C1);
\end{tikzpicture}
\caption{Two components, one with a provide, use and use-multiple ports, the second with a single provide port}
\label{fig:ports}
\end{center}
\end{figure}

In the rest of this paper, when a required service of a \emph{use} (or \emph{use-multiple}) port is filled and linked to a \emph{provide} port in the comonent assembly, only the use port stay visible, as illustrated in Figure~\ref{fig:assembly}.

\begin{figure}[h!]
\begin{center}
\begin{tikzpicture}[shorten >=1pt, node distance=2cm, on grid, auto]
   \node[component] (C) at (0,0) {$Comp_0$};
   \node[provide] (p) at (-1.5,0) {};
   \node[use] (u) at (1.5,0) {};
   \node[use] (um) at (0,1) {$m$};
   \node[component,right=of u] (C1) {$Comp_1$};
 
  \path[-]
    (p) edge node {} (C)
    (C) edge node {} (u)
    	edge node {} (um)
    (u) edge node {} (C1);
\end{tikzpicture}
\caption{Component assembly of Figure~\ref{fig:ports}}
\label{fig:assembly}
\end{center}
\end{figure}

Many component models exist, each of them with its own specifications and functionnalities, like CCM~\cite{corba:omg06} (CORBA Component Model), GCM~\cite{Baude} (Grid Component Model) or CCA~\cite{Armstrong:1999:TCC:822084.823232} (Common Component Architecture) for example. However, conepts of component, port, interface and assembly are common to many component models. The rest of this paper will only use those concepts to present the MS Language and its compiler whatever the component model is.

